\documentclass[a4paper, 13pt]{scrreprt}
\usepackage[utf8]{inputenc} 
\usepackage{amssymb}
\usepackage[ngerman]{babel}
\usepackage{amsthm}
\usepackage{amsmath}
\usepackage{graphicx}
%\usepackage{pstricks}
\usepackage[T1]{fontenc}
\usepackage{color}
\usepackage{lmodern}
\usepackage[pdfborder={0 0 0}]{hyperref}

\pagestyle{headings}

\newtheorem{satz}{Satz}[section]
\newtheorem{theorem}{Theorem}[section]
\newtheorem{lemma}[theorem]{Lemma}
\newtheorem{proposition}[theorem]{Proposition}
\newtheorem{corollar}[theorem]{Corollar}
\theoremstyle{definition} \newtheorem{definition}{Definition}[section]

\newenvironment{beweis}[1][Beweis]{\begin{trivlist}
\item[\hskip \labelsep {\bfseries #1}]}{\end{trivlist}}
\newenvironment{beispiel}[1][Beispiel]{\begin{trivlist}
\item[\hskip \labelsep {\bfseries #1}]}{\end{trivlist}}
\newenvironment{bemerkung}[1][Bemerkung]{\begin{trivlist}
\item[\hskip \labelsep {\bfseries #1}]}{\end{trivlist}}

\DeclareMathOperator*{\argmin}{arg\,min}

\newcommand{\RR}{\mathbb{R}}

\begin{document}
\title{Nichtlineare Dynamik}
\publishers{\small{Fehler in der Mitschrift an
\url{alexander.book@gmx.de}} oder 
\url{dominik.o@gmx.net}}
\maketitle
\tableofcontents


\chapter{Grundlagen}
\section{Dynamische Systeme}
\begin{definition}[dynamische Systeme]
Wir behandeln zwei Arten von dynamischen Systemen:
\begin{enumerate}
\item \emph{kontinuierliches dynamisches System}: Es gibt eine kontinulierliche Zeitvariable $t\in\mathbb{R}$
\item \emph{diskretes dynamisches System}: Es gibt eine kontinulierliche Zeitvariable $t\in\mathbb{Z}$
\end{enumerate}
Im folgenden bezeichnet $T$ entweder $\mathbb{R}$ oder $\mathbb{Z}$, je nachdem, welches dynamische System im Kontext verwendet wird.

Es gibt einen \emph{(Zustands-)Phasenraum} $X$, der den Zustand eines Systems mit verschiedenen Gr"o"sen beschreibt $(X\subseteq \mathbb{R}^n, n\in \mathbb{N})$. $x\in X$ beschreibt somit einen m"oglichen Zustand eines \emph{dynamischen Systems}. Falls $\operatorname{dim}(X) < \infty$, so nennt man es \emph{endlich dimensionales dynamisches System}. Andernfalls ($\operatorname{dim}(X) = \infty$) nennt man es \emph{unendlich dimensionales dynamisches System}. Mit \emph{Dynamik} bezeichnet man die zeitliche Ver"anderung des Zustands eines dynamischen Systems.
\end{definition}


Generell beginnt ein dynamisches System bei einer Anfangszeit $t_o$ und einem Zustand $x(t_0) = x_0 \in X$. Anhand dieses Punktes wird jedem andern Zeitpunkt ein eindeutiger Zustand zugewiesen ($x(t_0) = x_0 \Rightarrow \forall t\in T\  \exists! x_t\in\mathbb{R}^n: x(t) = x_t$)
Diese Zuordnung wird durch die \emph{Flussabbildung} definiert:
$$\phi\colon \mathbb{R}\times X\to X, \ \forall t \in T: x(t):= \phi(t-t_0, x_0) $$


\begin{definition}[Klassifikation von dynamischen Systemen]
Man unterscheidet dynamische Systeme in lineare und nicht-lineare Systeme:
\begin{enumerate}
\item \emph{Lineares dynamisches System}: $\phi(t, \cdot)\colon X \to X$ ist linear. Man schreibt dann auch $\phi(t, x) = \phi(t)x$. Dabei ist $\phi(t)$ ein linearer Operator f"ur alle $t\in T$
\item \emph{Nichtlineares dynamisches System}: $\phi(t, \cdot)\colon X \to X$ ist nicht linear.
\end{enumerate}
\end{definition}
\begin{definition}[Phasendiagramm]
Durch ein dynamischen Systems $(X,\phi)$ wird jedem Zustand $x\in X$ ein \emph{Orbit} zugeordnet:
$$\Gamma_x := \left \{\left. y \in X \right | \exists t\in T: \phi(t,x) = y\right \}$$ 
Ein \emph{Phasendiagramm} ist die Skizze des Orbits $\Gamma_x$ f"ur einige $x \in X$.
\begin{description}
\item[Bemerkung]Durch jeden Punkt $x\in X$ verl"auft genau ein Orbit $\Gamma_x$. Insbesondere k"onnen sich Orbits nicht traversal (selbst) schneiden.
\end{description}
\end{definition}
\subsection{Eigenschaften der Flussabbildung $\phi$}
Die Flussabbildung gen"ugt folgenden Eigenschaften:
\begin{enumerate}
\item $\forall x\in X: \phi(0,x) = x$
\item $\phi(\cdot, x)$ ist stetig f"ur alle $x\in X$.
\item $\phi(t, \cdot)$ ist stetig f"ur alle $t\in T$.
\item $\phi(t, \cdot)\colon X \to X$ ist ein Hom"oomorphismus (d.h. bijektiv und Umkehrabbildung ist stetig)
\item $\phi(s+t, x) = \phi\left(s, \phi(t,x)\right)$ f"ur alle $s,t \in T,\  x\in X$
\end{enumerate}

\section{Elementarste Typen von dynamischen Systemen}
Dynamische Systeme k"onnen auch implizit angegeben werden. Im Folgenden werden die zwei wichtigsten dynamischen Systeme f"ur diese Vorlesung vorgestellt.
\subsection{Gew"ohnliche Differentialgleichungs Systeme (GDG-Systeme)}
GDG-Systeme sind ein Beispiel f"ur kontinuierliche dynamische Systeme. Betrachtet man eine autonome gew"ohnliche Differentialgleichung 1. Ordnung
$$\dot x= v(x)$$
wobei $v\colon\mathbb{R}^n\to\mathbb{R}^n$ ein Vektorfeld ist. Durch das zugeh"orige AWP $x(0) = x_0$ wird die L"osung $x(t) = \phi(t, x_0)$ festgelegt, falls $v$ hinreichende Struktur besitzt. Falls $v$ beispielsweise lokal Lipschitz-stetig ist, liefert Picard-Lindel"of eine lokal eindeutige L"osung. 
Dies induziert ein dynamisches System $(X, \phi)$, wobei $X = \mathbb{R}^n$, bzw. $X$ das Definitionsgebiet des Vektorfeldes ist.
\begin{lemma}
Die durch dieses AWP induziert $\phi$ gen"ugt den Eigenschaften einer Flussabbildung
\end{lemma}
\begin{beweis}
Sei $\phi(t, x)$ die \emph{Fundamentall"osung} der Differentialgleichung
$$\dot x = v(x)$$
wobei $v\in C^1(\mathbb{R}^n)$. D.h. $x(t) = \phi(t,x)$ ist die eindeutige L"osung des zugeh"origen AWP $x(0) = x_0$.
Folglich ist $\phi(t+s, x)$ eine L"osung der Differentialgleichung f"ur alle $s\in \mathbb{R}$, denn:
$$\frac{\mathrm d}{\mathrm dt} \phi(t+s, x_0) = v\bigl(\phi(t+s, x_0)\bigr)$$
Aber $\left . \phi(t+s, x_0) \right |_{t=0} = \phi(s, x_0)$ ist die Anfangsbedingung dieser L"osung. Also l"ost $\phi(t+s, x_0)$ das AWP $x(0) = \phi(s, x_0)$.
Deswegen gilt $\phi(t+s, x_0) = \phi\left(t, (\phi(s, x_0)\right)$
\end{beweis}

\subsection{Hom"oomorphismus Systeme (Hom-Systeme)}
Betrachte einen Hom"oomorphismus $\psi\colon X \to X$. Dieser induziert ein diskretes dynamisches System wie folgt:
\begin{align*}
\phi(k, x) := \begin{cases}
\psi^k(x), &\mbox{ falls } k \in \mathbb{N} \\
\psi^0(x) = x,& \mbox{ falls } k = 0 \\
\psi^{-k}(x) := (\psi^{-1})^{-k}(x), &\mbox{ falls } k \in \mathbb{Z}\setminus\mathbb{N}_0
\end{cases}
\end{align*}
$\phi$ ist damit die Flussabbildung eines diskreten dynamischen Systems $(X, \phi)$.
\section{Gleichgewichtspunkte}
\begin{definition}
Ein Punkt $x_G \in X$ hei"st \emph{Gleichgewichtszustand(-punkt)} des dynamischen Systems $(X,\phi)$, falls gilt
$$ \forall t \in T: \phi(t, x_G) = x_G$$
\end{definition}

\subsection{Gleichgewichtspunkte in GDG-Systemen}
Sei $x_G$ ein Gleichgewichtspunkt des durch die Differentialgleichung $\dot x = v(x)$ induzierte dynamischen Systems. Dann gilt:
$$ \forall t\in \mathbb{R}: \phi(t, x_G) = x_G $$
Differenzieren liefert 
$$ \frac{\mathrm d}{\mathrm dt} \phi(t, x_G) = 0 $$
Somit liegt jeder Gleichgewichtspunkt des dynamischen Systems in der Nullstellenmenge des Vektorfeldes $v$.
$$ x_G \mbox{ Gleichgewichtspunkt } \Leftrightarrow x_G \in v^{-1}(\{0\}) $$

\subsection{Gleichgewichtspunkte in Hom-Systemen}
Sei $\psi$ ein Hom"oomorphismus. Sei $(X, \phi)$ das durch $\psi$ induzierte dynamische System. Somit muss f"ur jeden Gleichgewichtspunkt $x_G$ des dynamischen Systems gelten:
$$ \forall k\in\mathbb{Z}: \phi(k, x_G) = \psi^k(x_G) = x_G$$
F"ur $k=1$ folgt $x_G= \psi(x_G)$. Also sind alle Gleichgewichtspunkte des dynamischen Systems Fixpunkte von $\psi$. 
$$ x_G \mbox{ Gleichgewichtspunkt } \Leftrightarrow x_G \mbox{ Fixpunkt von } \psi $$

\subsection{Gleichgewichtspunkte von linearen dynamischen Systemen}
Im linearen Fall ist f"ur beide Typen GDG- bzw. Hom-Systeme ein trivialer Gleichgewichtspunkt $x_G = 0$ gegeben.
\begin{enumerate}
\item GDG-System: Gegeben sei die Differentialgleichung $$\dot x = v(x) = Ax, \ A \in \mathbb{R}^{n\times n}, \ x\in \mathbb{R}^n$$
Dann ist die Flussabbildung gegeben durch $\phi(t, x) = \exp{(tA)}x$. Zur Wiederholung: Die exponential Matrix ist definiert durch 
$\exp{(A)} = \sum_{k=0}^{\infty} \frac{A^k}{k!}$ und konvergiert f"ur jedes $A\in\mathbb{R}^{n\times n}$ gleichm"a"sig.

Die Bedingung ein Gleichgewichtspunkt zu sein ist $\phi(t, x) = 0$. Also erf"ullt $x_G = 0$ trivialer weise dieser Bedingung.

\item Hom-System: Sei $\psi$ eine lineare Funktion, also 
$$\psi( x) = Ax, \ A\in\mathbb{R}^{n \times }, \ x\in \mathbb{R}^n$$
Damit $\psi$ ein Hom"oomorphismus wird, muss $\det{(A)} \not = 0$ gelten. Die Bedingung f"ur ein Gleichgewichtspunkt ist diesesmal 
$$ \psi(x) = x $$
$x_G = 0$ erf"ullt dies Bedingung und ist daher ein Gleichgewichtspunkt.

\end{enumerate}

\subsection{Beispiele von Gleichgewichtspunkten}
\begin{beispiel}[Gleichgewichtspunkte des DGD-Systems]
Betrachte die Differentialgleichung $\dot x = x - x^3 = v(x), \ x \in \mathbb{R} = X$
Die Gleichgewichtspunkte sind also gegeben durch 
\begin{align*}
v(x) &= x - x^3 = 0 \\
& = x(1-x^2) = 0\\
\Rightarrow x_G^1 = 0, x_G^{2/3} = \pm 1
\end{align*}
\end{beispiel}

\begin{beispiel}[Gleichgewichtspunkte des Hom-Systems]
Betrachten den Hom"oomorphismus $\psi(x) = x^3,\ x\in \mathbb{R}$. Die Gleichgewichtspunkte des von $\psi$ induzierten dynamischen Systems sind gegeben durch
\begin{align*}
\psi(x) = x \Leftrightarrow x^3 = x &\Leftrightarrow x^3- x = 0\\
& x_G^1 = 0, x_G^{2/3} = \pm 1
\end{align*}

\end{beispiel}

\section{Dynamische Stabilit"at von Gleichgewichtspunkten im Sinne von Lyapunov}
Sei $(X, \phi)$ ein dynamisches System, $x_G\in X$ ein Gleichgewichtspunkt, $(X, d)$ ein metrischer Raum.

Wiederholung: $d$ hei"st Metrik auf $X$, falls $d\colon X \times X \to \mathbb{R}$ und f"ur beliebige Elemente $x, y, z\in X$ gilt:
\begin{enumerate}
\item $d(x,y) \geq 0, \ d(x, y) = 0 \Leftrightarrow x = y$ (Definitheit)
\item $d(x,y) = d(y, x)$ (Symmetrie)
\item $d(x, y) \leq d(x,z) + d(z, y) $ (Dreiecksungleichung)
\end{enumerate}

\begin{definition}
Ein Gleichgewichtspunkt $x_G$ hei"st
\begin{itemize}
\item \emph{stabil (im Sinne von Lyapunov)}, falls 
$$ \forall \varepsilon > 0 \ \exists \delta > 0 \ \forall x \in X, t \in T, t > 0: d(x, x_G) < \delta \Rightarrow d\left(\phi(t, x), x_G\right) < \varepsilon$$
\item \emph{instabil (im Sinne von Lyapunov)}, falls $x_G$ nicht stabil ist.
\item \emph{asymptotisch stabil (im Sinne von Lyapunov)}, falls $x_G$ stabil ist und gilt
$$ \exists b > 0\ \forall x \in X: d(x, x_G) < b \Rightarrow \lim_{t\to\infty}{d\left(\phi(t, x), x_G\right)} = 0$$
\end{itemize}
\end{definition}
\begin{figure}[htpb]
\centering
\includegraphics[width=0.45\textwidth]{img/stabilitaet/stabilitaet_lypunov.pdf}
\includegraphics[width=0.45\textwidth]{img/stabilitaet/instabilitaet_lypunov.pdf}
\caption{Stabilit"at(links); Instabilit"at (rechts)}
\end{figure}

\subsection{Indirekte Methode von Lyapunov}
\subsubsection{Indirekte Methode von Lyapunov f"ur GDG-Systeme}
Sei $v$ ein $C^1$-Vektorfeld ($v\in C^1(\mathbb{R}^n, \mathbb{R}^n)$), $x_G$ ein Gleichgewichtspunkt des von $v$ erzeugten GDG-Systems. Es bezeichne $\sigma(A)$ die Menge aller Eigenwerte der Matrix $A$.
\begin{lemma}
Betrachte die Jacobi-Matrix $J_v(x)$ an der Stelle $x = x_G$.
\begin{itemize}
\item Falls $\forall \lambda \in \sigma(J_v(x_G)): \operatorname{Re} \lambda < 0$, dann ist $x_G$ asymptotisch stabil.
\item Falls $\exists \lambda \in \sigma(J_v(x_G)): \operatorname{Re} \lambda > 0$, dann ist $x_G$ instabil.
\item Falls $v$ ein lineares dynamisches System induziert und es gilt 
$$ \forall \lambda \in \sigma(J_v(x_G)): \operatorname{Re} \leq 0 \mbox { und } \lambda \mbox{ halb einfach, falls } \operatorname{Re} \lambda = 0$$
dann ist $x_G$ stabil. Dabei ist ein Eigenwert $\lambda$ \emph{halb einfach}, falls seine geometrische Vielfachheit, seiner algebraischen Vielfachheit entspricht.
\end{itemize}
\end{lemma}
\subsubsection{Indirekt Methode von Lyapunov f"ur Hom-Systeme}
Sei $\psi $ ein $C^1$-Hom"oomorphismus ($C^1$-Diffeomorphismus), $x_G$ ein Gleichgewichtspunkt des von $\psi$ erzeugten Hom-Systems.
\begin{lemma}
Betrachte die Jacobi-Matrix von $\psi$ an der Stelle $x_G$
\begin{itemize}
\item Falls $\forall \lambda \in \sigma(J_\psi(x_G)): |\lambda| < 1$, dann ist $x_G$ asymptotisch stabil
\item Falls $\exists \lambda \in \sigma(J_\psi(x_G)): |\lambda| > 1$, dann ist $x_G$ instabil.
\item Falls $\psi$ ein lineares dynamisches System erzeugt und gilt 
$$\forall \lambda \in \sigma(J_\psi(x_G)): |\lambda| \leq 1 \mbox{ und } \lambda \mbox{ halbeinfach, falls } |\lambda| = 1$$
dann ist $x_G$ stabil.
\end{itemize}
\end{lemma}
\subsection{Direkte Methode von Lyapunov}
\subsubsection{Direkte Methode von Lyapunov f"ur GDG-Systeme}
Sei $v$ ein $C^1$-Vektorfeld, $x_G$ ein Gleichgewichtspunkt.
\begin{definition}
Eine \emph{(strikte) Lyapunov-Funktion} $V$ ist eine Funktion $V \in C^1(U, \mathbb{R})$, sodass $x_G \in U, \ U\subset \mathbb{R}^n$ offen und 
\begin{enumerate}
\item $V(x_G) = 0$
\item $\forall x \in U\setminus \{x_G\}: V(x) > 0$
\item $\forall x \in U: \langle \nabla V(x), v(x) \rangle \stackrel{(<)}\leq 0$ \\
				\(( \Rightarrow \partial_t V(\phi(t,x)) = \langle \nabla V(\phi(t,x)), v(\phi(t,x)) \rangle \stackrel{(<)}\leq 0 )\)
\end{enumerate}
\end{definition}

\begin{lemma}
Falls eine Lyapunov-Funktion f"ur $v$ um $x_G$ existiert dann ist $x_G$ stabil. Gilt strikte Ungleichheit in $(3)$, dann ist $x_G$ sogar asymptotisch stabil.
\end{lemma}

\begin{description}
\item[Bemerkung]
Falls \( U = \RR^2 \) und \(V\) eine strikte Lyapunov-Funktion zu \(x_G\), dann ist \(x_G\) global asymptotisch stabil.
\end{description}

\begin{beweis} 
Fall \("\leq" : \) \newline
Sei \(\varepsilon > 0\) hinreichend klein, sodass \(\overline{B_{\varepsilon} (x_G)} \subset U\). Sei \(m\) das Minimum von \(V\) auf \(\partial B_{\varepsilon} (x_G) \). Dies existiert, da \(\partial B_{\varepsilon} (x_G) \) kompakt und \(V\) stetig (Satz von Weierstra"s). Dann folgt mit Bedingung 1), 2) : \(m > 0\). \\
Definiere \( \tilde{U} := \{ x \in B_{\varepsilon} (x_G)\ |\ V(x) < m \} \not= \emptyset \) offen. (\(x_G \in \tilde{U}\) und insbesondere ex. \(\delta > 0 \) mit \(B_{\delta} (x_G) \subset \tilde{U} \), wie auch in jedem anderen Punkt von \(\tilde{U})\). \\
\(x_0 \in \tilde{U} \Rightarrow V(x_0) < m \) und damit \(V(\Phi(t,x_0)) \leq V(x_0) < m \)
\begin{addmargin}[38pt]{0pt}
	\(\Rightarrow \Phi(t,x_0) \notin \partial B_{\varepsilon} (x_G)\  \forall t \geq 0 \) \\
	\(\Rightarrow \Phi(t,x_0) \in B_{\varepsilon} (x_G) \) \\
	\(\Rightarrow x_G  \) ist Lyapunov-stabil
\end {addmargin}
\end{beweis}

\begin{beispiel} \(X = \RR^2\) 
	\[ \begin{cases} \dot{x} = y \\
			\dot{y} = x - x^3 \end {cases} 
	\]
\begin{itemize}
	\item Gleichgewichtspunkte: 
		\[v(x,y) = \left( \begin{array}{c} y \\ x-x^3 \end{array} \right) = \left(\begin{array}{c} 0 \\ 0 \end{array} \right) \]
		\[ \Leftrightarrow y = 0, x = 0 \ \lor x = \pm 1\]
		\[ \Rightarrow x_G^1 = \left(\begin{array}{c} 0 \\ 0 \end{array} \right),\  
			x_G^{2/3} = \left( \begin{array}{c} \pm 1 \\ 0 \end{array} \right ) 
		\]
	\item Konstruktion einer Lyapunov Funktion\\
		\( II \cdot y - I \cdot x \)
		\begin{align*}
			-x\dot{y} + y \dot{y} = -x^3y & = -x^3\dot{x} \\
			\Leftrightarrow \frac {d}{dt} \left( -0,5 x(t)^2 + 0,5 y(t)^2 + 0,25 x(t)^4 \right) & = 0 \\
			\Leftrightarrow -0,5 x(t)^2 + 0,5 y(t)^2 + 0,25 x(t)^4 & = C
		\end{align*}
		Dann ist 
			\[V(x,y) = -0,5 x(t)^2 + 0,5 y(t)^2 + 0,25 x(t)^4 - C\]
		eine Lyapunov-Funktion f"ur jedes \(x_G^i, (i = 1,2,3) \) bei geeigneter Wahl von \(C\), denn
		\begin{itemize}
			\item \(V(x_G^i) = 0 \) mit \( C = 0\) f"ur \( x_G^1\) und \( C = -0,25\) f"ur \(x_G^{2/3}\)
			\item \(\langle \nabla V(x,y), v(x,y) \rangle = 0\)\\
						\(\nabla V(x,y) = \left( \begin{array}{c} -x+x^3 \\ y \end{array} \right) = 
							\left( \begin{array}{c} 0 \\ 0 \end{array} \right) \)
			\item \(HV(x,y) = \left(\begin{array}{cc} -1+3x^2 & 0 \\ 0 & 1 \end{array}\right) \) \\
						\(HV(x_G^1) = \left(\begin{array}{cc} -1 & 0 \\ 0 & 1 \end{array} \right)\) indefinit \(\Rightarrow x_G^1\) ist Sattelpunkt von \(V\) \\
						\(HV(x_G^{2/3}) = \left(\begin{array}{cc} 2 & 0 \\ 0 & 1 \end{array}\right) \) pos. definit \(\Rightarrow x_G^{2/3} \) sind strikte lokale Minima von \(V \Rightarrow V > 0\) f"ur alle \( x \not= x_G^{2/3}\) in einer gewissen Umgebung von \(x_G^{2/3}\).\\
						\(\Rightarrow x_G^{2/3} \) sind Lyapunov-stabil.\\
						\(Jv(x,y) = \left( \begin{array}{cc} 0 & 1 \\ 1-3x^2 & 0 \end{array}\right)
						\Rightarrow Jv(x_G^1) = \left( \begin{array}{cc} 0 & 1 \\ 1 & 0 \end{array}\right) \Rightarrow \lambda_{1/2} = \pm 1 \) \\
						\( \Rightarrow Re(\lambda_{1/2}) > 0 \Rightarrow \) indirekte Methode: \(x_G^1\) ist instabil
			\end{itemize}
	
\end{itemize}
\end{beispiel}

\subsubsection{Direkte Methode f"ur Hom-Systeme} 
Direkte Methode von Lyapunov funktioniert entsprechend des GDG-Falls wobei in der Definition einer Lyapunov-Funktion die Bedingng 3) zu ersetzen ist durch: 
	\[\forall x \in U: V(\Psi(x)) \stackrel{(<)} \leq V(x)\  \]
wobei \(\Psi\) der erzeugende Hom"oomorphismus des Hom-Systems sei.


%-----------------------------------------------------------------------------------------------

\chapter{Lineare Systeme}

\section{GDG-Systeme}
Betrachte die Differentialgleichung
	\[\dot{x} = Ax =: v(x)\]
wobei $x\in\RR^n, A \in \RR^{n\times n}$ \emph{Systemmatrix}
\begin{satz}[Jordannormalform von A]
Es exisitiert eine invertierbare lineare Transformation $T:\RR^n \to \RR^n $, sodass 
$$ J = T ^{-1} A T$$
in Jordan-Normalform ist. Es gilt au"serdem
	\begin{align*}
	& e^{Jt} = e^{T^{-1}AT} = \sum_{j=1}^{\infty} \frac{t^j}{j!} (T^{-1}AT)^j = T^{-1} \sum_{j=1}^{\infty} \frac{t^j}{j!} A^j T = T^{-1} e^{At} T
	\end{align*}
	Dabei ist \(J\) die Matrix der Flu"sabbildung des \emph{J-Systems} \(\dot{\xi} = J\xi\), \(A\) die Matrix des \emph{A-Systems} \(\dot{x} = Ax\)
\end{satz}
	
\begin{description}
\item[Terminologie]
Man sagt, dass das \(J\)- und das \emph{A-System} bez"uglich der linearen Transformation \(T\) zueinander \emph{konjugiert} oder \emph{"aquivalent} sind.
\end{description}

\begin{description}
\item[Bemerkung]
 \(T\) bildet die Orbits des J-Systems bijektiv auf die Orbits des A-Systems ab. Sei dazu \(\xi \in \RR^n\). Dann gilt f"ur die Orbits durch $\xi$
\begin{align*}
	 & e^{Jt} \xi  = T^{-1}e^{At} T \xi \\
	\Leftrightarrow & Te^{Jt} \xi = e^{At}T\xi = e^{At}x
\end{align*}
\(T\) bildet den Orbit durch \(\xi\) des J-Systems  auf den Orbit durch \(x = T\xi\) des A-Systems  ab.
Daher klassifiziert man lineare Differentialgleichungen modulo einer linearen Transformation $T$.

\end{description}


\section{Klassifikation von Phasendiagrammen von GDG-Systemen f"ur $n=1$}
Die erzeugende Differentialgleichung lautet
	\[ \dot{x} = ax, \qquad a \in \RR \]
Man erh"alt dann folgende Klassifikation in Abh"anigkeit von a:
	\begin{enumerate}
		\item $a=0: $ alle Punkte sind Gleichgewichtspunkte
		\item $a > 0: x = 0$ ist eine Quelle
		\item $a > 0: x = 0 $ ist eine Senke
	\end{enumerate}
\section{Klassifikation von Phasendiagrammen von GDG-Systemen f"ur $n=2$}
	\[\dot{x} = Ax, \qquad A \in \RR^{2\times 2} \]
Die Jordannormalform von $A$ kann dann folgende $3$ Typen annehmen
\subsection{Jordannormalform ist in Diagonalform}
\(A\) habe Eigenwerte \(\lambda_1, \lambda_2 \in \RR\) halbeinfach. Die Jordannormalform von $A$ ist gegeben durch $$J = \left(\begin{array}{cc} \lambda_1 & 0 \\ 0 & \lambda_2 \end{array} \right) $$ Das dazugeh"orige Anfangswertproblem lautet dann 
			\[ \begin{cases} \dot{\xi_1} = \lambda_1 \xi_1 , \ \xi_1(0) = \xi_{10}\in\RR \\
				\dot{\xi_2} = \lambda_2 \xi_2 , \ \xi_2(0) = \xi_{20}\in \RR \end {cases} 
			\]
Die L"osung der obigen Differentialgleichung ist offensichtlich 
			\begin{align*}
						\xi_1(t)   = \xi_{10} e^{\lambda_1 t} \\
						\xi_2(t)   = \xi_{20} e^{\lambda_2 t} 
			\end{align*}
Nun wollen wir $\xi_2$ in Abh"anigkeit von $\xi_1$ angeben, falls alle Rechnungen so durchf"uhrbar sind:
					\begin{align*}
						&\frac{\xi_1}{\xi_{10}} = e^{\lambda_1t} \\
						\Leftrightarrow &\ln{\left(\frac{\xi_1}{\xi_{10}}\right)} = \lambda_1 t 		
						\Leftrightarrow t = 	
						\frac {1}{\lambda_1} \ln \left(\frac{\xi_1}{\xi_{10}} \right) \\
						\Rightarrow \xi_2 &  = \xi_{20} \exp\left( \frac{\lambda_2}{\lambda_1} \ln\left(	\frac{\xi_1}{\xi_{10}} \right) \right) = \xi_{20} \left(\frac{\xi_1}{\xi_{10}}\right)^{\frac{\lambda_2}{\lambda_1}}  					
				\end{align*}
Nun k"onnen die Phasendiagramme klassifiziert und skizziert werden. Es ergeben sich daher die F"alle
\subsubsection{1. Fall: \(0<\lambda_1 < \lambda_2\) }
	\( x = 0 \) wird \emph{instabiler Knoten 2. Art} genannt.
\subsubsection{2. Fall: \(\lambda_2 < \lambda_1 < 0\)}
	\( x = 0\) ist wird \emph{stabiler Knoten 2. Art} genannt.
	
	\begin{figure}[htpb]
		\centering
		\includegraphics[height=0.35\textheight]{img/lin_sys/lin_sys_1.pdf}
		\includegraphics[height=0.35\textheight]{img/lin_sys/lin_sys_2.pdf}
		\caption{1. Fall (links); 2. Fall (rechts)}
	\end{figure}	
	
\subsubsection{3. Fall: \( 0 < \lambda_1 = \lambda_2\) }
\( x= 0\) wird \emph{instabiler Knoten 1. Art} genannt.
\subsubsection{4. Fall: $\lambda_1 = \lambda_2 < 0$ }
\( x = 0\) wird \emph{stabiler Knoten 1. Art} genannt.
	\begin{figure}[htpb]
		\centering
		\includegraphics[height=0.20\textheight]{img/lin_sys/lin_sys_3.pdf}
		\includegraphics[height=0.20\textheight]{img/lin_sys/lin_sys_4.pdf}
		\caption{3. Fall (links); 4.Fall (rechts)}
	\end{figure}
\subsubsection{5. Fall: \(\lambda_1 < 0 < \lambda_2\)}
					\(x = 0\) wird \emph{Sattelpunkt} genannt und ist offensichtlich instabil. Es ergeben sich in diesem Fall als Orbits Hyperbeln.
		\begin{figure}[htpb]
		\centering
		\includegraphics[height=0.20\textheight]{img/lin_sys/lin_sys_5.pdf}
		\caption{5. Fall}
	\end{figure}
	
	
	
	
\subsection{Jordannormalform ist in Pseudo-Diagonalform} 
$A$ habe einen geometrisch einfachen und algebraisch doppelten Eigenwert \({\lambda \in \RR}\). Die Jordannormalform von $A$ ist dann gegeben durch
				$$ J = \left( \begin{array}{cc} \lambda & 1 \\ 0 & \lambda \end{array} \right) $$
Das dazuge"orige Anfangswertproblem lautet
				\[\begin{cases} 
					\dot{\xi_1} = \lambda \xi_1 + \xi_2,\  \xi_1(0) = \xi_{10} \in \RR \\
				 	\dot{\xi_2} = \lambda \xi_2, \qquad\  \xi_2(0) = \xi_{20} \in \RR
				\end{cases} \]
Die L"osungen sind schlie"slich folgenderma"sen gegeben

				$$
				\Rightarrow \xi_2(t) = \xi_{20} e ^{\lambda t} \qquad \Rightarrow \xi_1(t) = \xi_{10} e ^{\lambda t} + t \xi_{20} e ^{\lambda t}$$
				Die Orbits sind analog zur vorherigen Jordannormalform darstellbar als
				$$ \xi_1 = \left( \frac{\xi_{10}}{\xi_{20}} + \frac{1}{\lambda} \ln{\frac{\xi_2}{\xi_{20}}}\right) \xi_2$$
solange keine ung"ultige Rechenoperation durchgef"uhrt wird. 
	\begin{figure}[htpb]
		\centering
		\includegraphics[height=0.31\textheight]{img/lin_sys/lin_sys_6.pdf}
		\includegraphics[height=0.31\textheight]{img/lin_sys/lin_sys_6_2.pdf}
		\caption{1. Fall (links); 2.Fall (rechts)}
	\end{figure}
\subsubsection{1. Fall: $\lambda < 0$} 
$x = 0$ wird \emph{stabiler Knoten 3. Art} genannt.
\subsubsection{2. Fall: $\lambda < 0$}
$ x = 0 $ wird \emph{instabiler Knoten 3. Art} genannt.

\subsection{Jordannormalform ist in keiner Diagonalform} 
$A$ habe ein paar komplex konjugierte Eigenwerte $\lambda_{1/2} = \alpha \pm i \beta $. Die reelle Jordannormalform von $A$ ist gegeben durch
				$$ J = \left( \begin{array}{cc} \alpha & \beta \\ - \beta & \alpha \end{array}\right) $$ 
und es ergibt sich das Anfangswertproblem
		\[\begin{cases} 
			\dot{\xi_1} =  \alpha \xi_1 + \beta\xi_2, \ \  \xi_1(0) = \xi_{10} \in \RR \\
			\dot{\xi_2} = -\beta \xi_1 + \alpha \xi_2 , \ \xi_2(0) = \xi_{20} \in \RR
		\end{cases} \]
Die L"osung ist daher		
		\[ \phi(t,\xi_0) = e^{Jt} \xi_0 = e^{(A+B)t} \xi_0 \]
				wobei 
				$$ A = \left( \begin{array}{cc} \alpha & 0 \\ 0 & \alpha \end{array}\right), \ B = \left( \begin{array}{cc} 0 & \beta \\ - \beta & 0 \end{array}\right) $$
Offensichtlich kommutieren $A$ und $B$ miteinander und es gilt $ e^{(A+B)t} = e^{At} e^{Bt} $. Berechnen wir nun die Exponentialmatrix von $A$ bzw. $B$ explizit, so erhalten wir
				$$ e^{At} = e^{\alpha t} \cdot I_2, \  e^{Bt} = \left(\begin{array}{cc} \cos{(\beta t)}  & \sin{(\beta t)} \\ - \sin{(\beta t )} & \cos{(\beta t })  \end{array}\right) \in SO(2) $$
Die explizite L"osung ist dann
				\[\phi(t, \xi_0) = e^{\alpha t} \underbrace{\left(\begin{array}{cc} \cos{(\beta t)}  & \sin{(\beta t)} \\ - \sin{(\beta t )} & \cos{(\beta t })  \end{array}\right)}_{Drehmatrix} \xi_0
				\]
				
\subsubsection{1. Fall: \(\alpha \not = 0\)}
$x=0$ wird \emph{Strudel(Wirbel)} genannt. Falls $\alpha < 0$ so sagt man zus"atzlich, dass $x$ stabil ist. F"ur $\alpha > 0$ entsprechend instabil.
\subsubsection{2. Fall: \(\beta \not = 0\)}
$x=0$ ist \emph{mit den Uhrzeigersinn orientiert}, falls $\beta < 0$. Entsprechend, falls $\beta > 0$ \emph{ gegen den Uhrzeigersinn orientiert}.
\subsubsection{3. Fall: \(\alpha = 0\)}
$x=0$ hei"st \emph{Zentrum}. Dieser ist stabil, jedoch nicht asymptotisch stabil.

\begin{figure}[htpb]
		\centering
		\includegraphics[height=0.25\textheight]{img/lin_sys/lin_sys_7_1.pdf}
		\includegraphics[height=0.25\textheight]{img/lin_sys/lin_sys_7_2.pdf}
		\caption{$\beta < 0 < \alpha$ (links); $\alpha < 0 < \beta$ (rechts)}

		\includegraphics[height=0.25\textheight]{img/lin_sys/lin_sys_7_3.pdf}
		\caption{$\alpha = 0, \beta < 0$}
	\end{figure}
\newpage
\section{Reduktion des Klassifikationsproblems}
\begin{definition}
	Sei \((X,\phi)\) ein dynamisches System. Dann hei"st
	\begin{itemize}
		\item \( M \subset X \) \emph{positiv invariant} \(\Leftrightarrow  \forall t \geq 0: \phi(t,M) \subset M \)
		\item \(\begin{aligned}[t] M \subset X  \text{ \emph{negativ invariant}} 
			& \Leftrightarrow \forall t \leq 0: M \subset \phi(t,M)  \\
			& \Leftrightarrow \forall t \geq 0: \phi(-t,M) \subset M \\
			& \Leftrightarrow \forall t \leq 0: \phi(t,M) \subset M  
			\end{aligned} \)
			
		\item \(\begin{aligned}[t] M \subset X \text{ \emph{invariant}} & \Leftrightarrow M\ \text{positiv und negativ invariant}\\
			& \Leftrightarrow \forall t \in T: \phi(t,M) = M \ 
			\end{aligned} \)
		\end{itemize}	
		Ist \( M \subset X \) invariant, dann bildet 
			\( (M,\left. \phi(t,\cdot)\right|_M) \) ein dynamisches System auf M und wird \emph{Teilsystem} des urspr"unglichen Systems \((X,\phi)\) genannt. 	
\end{definition}

\begin{description}
\item [Bemerkung]
	Jeder invariante Untervektorraum \(U \subset \RR^n \) bzgl. der linearen Abbildung 
		\[x \mapsto Ax : \RR^n \rightarrow \RR^n \]
(d.h. $x\in U\Rightarrow Ax \in U$ ) ist ein invarianter Untervektorraum des GDG-Systems \(\dot{x} = Ax\), denn
		\[\phi(t,x_0) = e^{At} x_0 = \sum_{j=0}^{\infty} \frac{t^j}{j!} \underbrace{A^j x_0}_{\in U}, \qquad x_0 \in U\]
Der Wert der Summe liegt in \(U\), da $U$ abgeschlossen und sie Grenzwert ist von
	\[ e^{At} x_0 = \lim_{N\to \infty} \underbrace{\sum_{j=0}^{N} \frac{t^j}{j!} A^j x_0 }_{\in\  U\ \forall N} \]
\end{description}

\begin{corollar}
	Alle Eigenr"ame \(E_j\) (bzw. verallgemeinerte Eigenr"aume), sowie deren direkte Summen sind kanonisch invariante Unervektorr"aume des Systems
		\[\dot{x} = Ax, \qquad A \in \RR^{n\times n} \]
	\underline{Speziell:} Ist \(\RR^n = \oplus_{j=1}^{N} E_j \) eine direkte Summe von (relativ niedrig dimensionierten) Eigenr"aumen von \(A\), dann ist das urspr"ungliche System \(\dot{x}=Ax\) das direkte Produkt der Teilsysteme auf den \(E_j\).
	Falls sich die Teilsysteme vollst"andig analysieren bzw. klassifizieren lassen, dann auch das urspr"ungliche System \(\dot{x} = Ax\) im \(\RR^n\)
\end{corollar}

\begin{definition}
	Spezielle (verallgemeinerte) Eigenr"aume von \(A\) und damit invariante Untervektorr"aume von \(\dot{x}=Ax\) : 
	\begin{itemize} 
		\item stabiler Unterraum von \(\dot{x}=Ax\)
			$$E^s := \left\{\left. v\in \RR^n \right| (A-\lambda \operatorname{id})(v) = 0 \land \operatorname{Re}(\lambda) < 0\right\}$$ 
			Dies ist der verallgemeinterte Eigenraum zu allen Eigenwerten $\lambda$ von $A$ mit $\operatorname{Re}\lambda < 0$.
		\item instabiler Unterraum von \(\dot{x}=Ax\)
		$$E^u := \left\{\left. v\in \RR^n \right| (A-\lambda \operatorname{id})(v) = 0 \land \operatorname{Re}(\lambda) > 0\right\}$$ 
			Dies ist der verallgemeinterte Eigenraum zu allen Eigenwerten $\lambda$ von $A$ mit $\operatorname{Re}\lambda > 0$.
		\item Zentrums-Unterraum von \(\dot{x}=Ax\)
		$$E^c := \left\{\left. v\in \RR^n \right| (A-\lambda \operatorname{id})(v) = 0 \land \operatorname{Re}(\lambda) = 0\right\}$$ 
			Dies ist der verallgemeinterte Eigenraum zu allen Eigenwerten $\lambda$ von $A$ mit $\operatorname{Re}\lambda = 0$.
	\end{itemize}
\end{definition}
\begin{figure}[htpb]
		\centering
		\includegraphics[height=0.30\textheight]{img/eigenraeume.pdf}
		\caption{$E^c$ entscheidet viel "uber das Verhalten der Orbits}
\end{figure}

\begin{satz}
Es gilt: 
	\[\RR^n = E^s \oplus E^u \oplus E^c \]
\end{satz}

\begin{description}
	\item[Terminologie]Spezielle Eigenraum-Typen des GDG-Systems $\dot x = Ax$ \\
	\begin{itemize}
		\item \(E^c = \{0\} \Rightarrow x =0\) hei"st hyperbolischer Gleichgewichtspunkt 
		\item \(E^c = \{0\}, {E^s \not= \{0\}, E^u \not= \{0\}} \Rightarrow x =0\) hei"st Sattelpunkt
		\item \(E^c = \{0\}, E^u = \{0\} \Rightarrow x =0\) hei"st Senke (asympt. stabil) 
		\item \(E^c = \{0\}, E^s = \{0\} \Rightarrow x =0\) hei"st Quelle (instabil)
		
	\end{itemize}
\end{description} 

\section{Klassifikation von Phasendiagrammen von Hom-Systemen f"ur $n=1$}

Sei $X = \RR, \ \psi\colon X \to X$ ein linearer Hom"omorphismus, der das lineare dynamische Systeme $(X,\phi)$ erzeugt. Insbesondere ist $\psi(x) = ax$ f"ur ein $a\in\RR\setminus\{0\}$.

Man kann dann die Orbits folgenderma"sen klassifizieren.
\subsubsection{Falls $|a| < 1$}
$x=0$ wird \emph{Senke} genannt und ist stabil.
\subsubsection{Falls $|a| > 1$}
$x=0$ wird \emph{Quelle} genannt und ist instabil.
\subsubsection{Falls $ a < 0$}
$x=0$ wird \emph{orientierungsumkehrend} genannt.
\subsubsection{Falls $a > 0$}
$x=0$ wird \emph{orientierungserhaltend} genannt.
\subsubsection{Falls $|a| = 1$}
$x=0$ wird \emph{Zentrum} genannt. Ist $a = 1$, so ist jeder Punkt $x\in\RR$ ein Gleichgewichtspunkt. F"ur $a=-1$ ergeben sich 2-periodische Orbits (gez"ahlt an der minimalen positiven Periode).

\begin{figure}[htpb]
		\centering
		\includegraphics[width=1\textwidth]{img/lin_sys/diskret/lin_sys_1.pdf}
		\caption{$|a| < 1, \ a < 0$}

		\includegraphics[width=1\textwidth]{img/lin_sys/diskret/lin_sys_2.pdf}
		\caption{$|a| > 1, \ a > 1$}

		\includegraphics[width=1\textwidth]{img/lin_sys/diskret/lin_sys_3.pdf}
		\caption{$a = -1$}
\end{figure}

\begin{description}
\item[Bemerkung]
	Jeder der bzgl. der linearen Abbildung \(x\mapsto Ax \) invarianter Unterverktorraum \(U\) ist invariant bzgl. des von \(\psi(x) = Ax \) erzeugten dynamische Systems.
\end{description}

%-----------------------------------------------------------------------------------------------

\chapter{Grobman-Hartman-Theorem}

\section{Kontinuierlicher Fall}
Sei $(X,\phi)$ ein dynamisches System, das durch die Differentialgleichung ${\dot x = v(x)}$ induziert ist, wobei $v \in C^k(\RR^n, \RR^n), \ k \geq 1$. Sei zus"atzlich  $x_G$ ein Gleichgewichtspunkt des dynamischen Systems. Betrachte die \emph{Linearisierung} des Systems um $x_G$ 
\[\dot{\xi}=Jv(x_G)\xi, \ \xi =x-x_G\] 
( $\dot{\xi}(x_G)\approx v(x)$, falls $\Vert\xi\Vert\ll 1$ )

\begin{satz}[Grobman-Hartman]\label{Grobman-Hartman}
Gegeben sei ein dynamisches System $(X,\phi)$ wie oben, wobei $x_G$ ein hyperbolischer Gleichgewichtspunkt ist, d.h. $\operatorname{Re}\lambda \neq 0$ f"ur alle Eigenwerte $\lambda$ von $Jv(x_G)$. Dann existiert eine Umgebung $U\subseteq \RR^n$ von $\xi =0$ und ein Hom"oomorphismus $h:U\to\mathbb{R}^n$, so dass \[ \forall t\in D: h(e^{Jv(x_G)t}\xi)=\phi(t,h(\xi))\ \] wobei $D := \left \{ \left . t \in \RR \right| e^{Jv(x_G)t}\xi \in U \right \}$ bezeichne.
\end{satz}
\begin{figure}[htpb]
		\centering
		\includegraphics[width=1\textwidth]{img/grobman-hartman.pdf}
		\caption{Illustration Grobman-Hartman-Theorem}
\end{figure}

Somit bildet $h$ hom"oomorph die Orbits des linearisierten Systems durch $\xi\in U$ auf diejenigen des nichtlinearen Systems durch $h(\xi)$ ab, wobei die zeitliche Orientierung erhalten bleibt. Man sagt, die beiden Systeme sind mittels des Hom"oomorphismus \emph{topologisch konjugiert} zueinander. Insbesondere ist damit also das lokale Phasenportrait des nichtlinearen Systems nahe $x_G$ ein hom"oomorphes Abbild des lokalen Phasenportraits des linearisierten Systems in U; die Bezeichnung zur Typisierung (Klassifikation) entsprechender hyperbolischer Gleichgewichtspunkte nichtlinearer Systeme "ubernimmt man vom linearen Fall, z.B: Ist $\xi =0$ ein Sattelpunkt von $\dot{\xi}=Jv(x_G)\xi$, dann ist auch $x_G$ ein Sattelpunkt von $\dot{x}=v(x)$.

\begin{description}
\item[Bezeichnung]Wir f"uhren folgende Bezeichnungen ein
\begin{itemize}
\item $h(E^s\cap U)=: W_{loc}^s(x_G)$ lokale stabile Mannigfaltigkeit von $x_G$ (positiv invariant)\\
\item $h(E^u\cap U)=: W_{loc}^u(x_G)$ lokale instabile Mannigfaltigkeit von $x_G$ (negativ invariant)\\
\item $W^s(x_G):=\{x\in\mathbb{R}^n|\lim\limits_{t\to +\infty}\phi (t,x)=x_G\}$ hei"st (globale) stabile Mannigfaltigkeit von $x_G$\\
\item $W^u(x_G):=\{x\in\mathbb{R}^n|\lim\limits_{t\to -\infty}\phi (t,x)=x_G\}$ hei"st (globale) instabile Mannigfaltigkeit von $x_G$
\end{itemize}
\end{description}

\begin{bemerkung}
$W^s(x_G)$ und $W^u(x_G)$ sind invariant, d.h.
\begin{align*}
\phi(t,W^{s/u}(x_G))=W^{s/u}(x_G)\ \forall t\in\mathbb{R} \\
x\in W^s(x_G)\ \Rightarrow\ \lim_{t\to\infty}\phi(t,x)=x_G\\
\Rightarrow\ \lim_{t\to\infty}\phi(t,\phi(s,x))=\lim_{t\to\infty}\phi(t+s,x)=x_G \text{ f"ur jedes }  s\in\mathbb{R}\\
\Rightarrow\ \phi(s,x)\in W^s(x_G)\\
\Rightarrow\ \phi(s,W^s(x_G))=W^s(x_G)\ \forall s\in\mathbb{R}
\end{align*}
\end{bemerkung}

\begin{satz}["Uber die lokalen stabilen und instabilen Mannigfaltigkeiten eines hyperbolischen Gleichgewichtspunktes]
Unter den Voraussetzungen von \eqref{Grobman-Hartman} gibt es eine Umgebung $U\subseteq \RR^n$ von $x_G$, sodass  Abbildungen
\[h^s:E^s\cap V\to E^u \text{ und } h^u:E^u\cap V\to E^s\]
existieren, die so glatt sind wie das Vektorfeld $v(x)$, so dass
\[W_{loc}^s(x_G)=\operatorname{graph}(h^s, E^s\cap V) \]und
\[W_{loc}^u(x_G)=\operatorname{graph}(h^u, E^u\cap V) \]
mit $h^{s/u}(x_G)=0$ und $J_{h^{s/u}}(x_G)=0$, d.h. $W_{loc}^{s/u}(x_G)$ ist in $x_G$ tangential zu $E^{s/u}$. Speziell kann $V = h(U)$ gew"ahlt werden, wobei $h$ der Hom"oomorphismus aus \eqref{Grobman-Hartman} ist.
\end{satz}

\begin{beispiel}
Gegeben sei folgende Differentialgleichung
\begin{align*}
v( \begin{pmatrix} x \\ y\end{pmatrix} ) = \begin{cases}
&\dot{x}=x\\
&\dot{y}=-y+x^2
\end{cases}
\end{align*}
Ein Gleichgewichtpunkt ist $x_G=(0,0)$. Die Jacobi-Matrix erf"ullt in $x_G$ 
\[Jv(x_G)=\begin{pmatrix} 1&0\\ 0&1 \end{pmatrix}\]
Daher sind die Eigenwerte 
\[ \lambda_1=-1, \lambda_2=1\\
\Rightarrow\ x_G  \text{ hyperbolischer Sattelpunkt}\]
und der Satz von Grobman-Hartman ist anwendbar. Die Orbitgleichung erh"alt man folgenderma"sen:
\begin{align*}
\frac{dy}{dx}=\frac{\frac{dy}{dt}}{\frac{dx}{dt}}=\frac{-y+x^2}{x} \text{ (f"ur $x\neq 0$) } =-\frac 1 x \cdot y+x\\
\Rightarrow\ y(x)=\frac 1 3 x^2+\frac c x, c\in\mathbb{R} \text{ beliebig} \\
\Rightarrow\ h^u:\mathbb{R}\to\mathbb{R}, x\mapsto\frac{x^2} 3\ (c=0)\\
\Rightarrow\ h^u(0)=0, (h^u)'(0)=Jh^u(0)=0\\
h^s:\mathbb{R}\to\mathbb{R}, x\mapsto 0
\end{align*}
\end{beispiel}


\section{Diskreter Fall}
Sei $\psi$ ein $C^k$-Diffeomorphismus von $\mathbb{R}^n$ nach $\mathbb{R}^n$, d.h. $\psi$ ist bijektiv und $\psi^{-1}\in C^k(\mathbb{R}^{n},\mathbb{R}^{n})$, $x_G$ ein Gleichgewichtspunkt des von $\psi$ erzeugten dynamischen Systems. Betrachte die Linearisierung dieses Systems in $x_G$, erzeugt durch $J\psi(x_G)$ (regul"ar). ($\psi(x)\approx J\psi(x_G)\xi, \xi=x-x_G, \Vert\xi\Vert\ll 1$)

\begin{satz}[Grobman-Hartman]
Unter diesen Voraussetzungen existiert eine Umgebung $0\in U\subset \mathbb{R}^n$ und ein Hom"oomorphismus $h:U\to h(U)\subset\mathbb{R}^n, h(0)=x_G$, sodass das von $J\psi(x_G)\xi$ erzeugte System bzgl. $h$ lokal topologisch konjugiert ist, d.h.
\begin{align*}
h(J\psi(x_G)\xi)=\psi(h(\xi)), \xi\in U\\
h(J\psi(x_G)^k\xi)=\psi^k(h(\xi)), k\in\mathbb{Z} \text{ beliebig}
\end{align*}
sofern $x_G$ ein hyperbolischer Gleichgewichtspunkt des $\psi$-Systems ist, d.h. $\xi=0$ ein hyperbolischer Gleichgewichtspunkt des linearisierten $J\psi(x_G)$-Systems ist.
\end{satz}

Die restliche Grobman-Hartman-Theorie ist analog zum kontinuierlichen Fall.

\begin{beispiel}
$\psi(x)=x^3$ erzeugender Hom"oomorphismus ($C^k$-Diffeomorphismus f"ur $1\leq k\leq\infty$ au"serhalb von $x=0$)\\
Als Voraussetzung der Grobman-Hartman-Theorie gen"ugt es, wenn die Voraussetzungen lokal nahe der betrachteten Gleichgewichtspunkte erf"ullt sind.
\[x_G=\pm 1, J\psi(x_G)=3>1\ \Rightarrow\ x_G\text{ orientierungserhaltende Quelle}\]
Gesucht ist ein Hom"oomorphismus $h$, welcher das $\psi$- und das $J\psi(x_G)$-System lokal nahe $x_G=\pm 1$ konjugiert (in $U_1=(-\infty, 0), U_2=(0,+\infty)$).\\
$h:\mathbb{R}\to\mathbb{R}, \xi\mapsto h(\xi)$ stetig, bijektiv, sodass
\begin{align*}
h(3\xi)=h(\xi)^3\ \forall\xi\in U\\
\Leftrightarrow \ln h(3\xi)=3\ln h(\xi)\\
\Rightarrow\ \ln\circ h(\xi)=\xi\\
\Rightarrow\ h(\xi)=e^{\xi}, h(0)=1 \text{ in } U_2=(0,+\infty)
\end{align*}
\end{beispiel}

\chapter{Periodische Orbits}
\section{Begriff und Bestimmung von periodischen Orbits}
\begin{definition}
Sei $(X,\phi)$ ein dynamisches System. Ein Orbit $\Gamma_{x_p} = \left \{ \left. \phi(t, x_p) \right | t \in \RR \right \}$ hei"st $T$-periodisch, falls $T>0$ und 
$$\forall t \in \RR: \phi(t, x_p) = \phi(t + T, x_p)$$
Das minimale $T>0$ hei"st \emph{Periode} des Orbits $\Gamma_{x_p}$. $x_p$ nennt man $T$-periodischen Punkt des Systems.
\end{definition}
\begin{figure}[htpb]
		\centering
		\includegraphics[width=0.8\textwidth]{img/periodische_orbits/disk_kont.pdf}
		\caption{periodische Orbits}
\end{figure}

\begin{description}
\item[Bemerkung]
Falls $x_p$ ein $T$-periodischer Punkt ist, so ist auch jeder andere Punkt $x\in \Gamma_{x_p}$ $T$-periodisch.
\end{description}
\subsection{Bestimmungsgleichung f"ur periodische Punkte}
Die Bestimmungsgleichung ist folgenderma"sen gegeben
$$\phi(T, x_p) = \phi(0, x_p)$$
f"ur ein minimales $T > 0$. Speziell im diskreten Fall ergibt sich
$$\phi(T, x_p) = \psi^T(x_p) = x_p$$

\begin{beispiel}
$\psi(x) = -x , \ x \in \RR$. 
Bestimmungsgleichung f"ur $2$-periodische Punkte
$$\psi^2(x_p) = \operatorname{id}(x_p) = x_p$$
Folglich ist jeder Punkt $x_p\in\RR\setminus\{0\}$ ein $2$-periodischer Punkt, also gilt ${\Gamma_{x_p} = \left \{ x_p, -x_p\right\}}$. F"ur $x = 0$ liegt ein Gleichgewichtspunkt vor (man sagt auch $1$-periodisch). 
\begin{figure}[htpb]
		\centering
		\includegraphics[width=0.7\textwidth]{img/periodische_orbits/beispiel_per_orbit.pdf}
		\caption{periodische Orbits}
\end{figure}
\end{beispiel}
\section{Poincar\'{e}  Abbildung f"ur GDG-Systeme}
Sei $(X,\phi)$ ein dynamisches System, das durch die Differentialgleichung $\dot x = v(x), \ x \in \RR^n$ erzeugt wird.
\begin{definition}
Sei $x_p$ ein $T$-periodischer Punkt. Es existiert ein $n\in\RR^n$, sodass $\langle v(x_p), n \rangle \neq 0$, beispielsweise $n = v(x_p)$. Die $(n-1)$-dimensionale Untermannigfaltigkeit
$$\Sigma_{x_p} := \left \{ \left. x \in X \right| \langle x - x_p, n \rangle =0 \right \}$$ 
schneidet den Orbit $\Gamma_{x_p}$ \emph{transversal} in $x_p$ und wird auch \emph{Poincar\'{e} Schnitt} genannt. Sei $V\subseteq \RR^n$ eine hinreichend kleine Umgebung von $x_p$. Die \emph{erste R"uckkehrzeit} $\tau\colon \Sigma_{x_p} \cap V \to \RR$ ist definiert als
$$\tau(x) := \min \left \{ \left. t> 0 \right| \phi(t, x) \in \Sigma_{x_p} \cap B_{\varepsilon(x)}(x) \right \}$$
wobei $\varepsilon(x)$ hinreichend klein gew"ahlt ist.
\end{definition}
\begin{description}
\item[Bemerkung] 
Die erste R"uckkehrzeit gibt die Zeit an, die ben"otigt wird um, ausgehend vom Punkt $x \in \Sigma_{x_p}\cap V$, die transversale Menge $\Sigma_{x_p}$ nach einem vollen Umlauf wieder zu schneiden. Das hei"st es gilt $\phi(\tau(x), x) \in \Sigma_{x_p}$, sowie $\tau(x_p) = T$ nach Definition.
\end{description}
\begin{figure}[htpb]
		\centering
		\includegraphics[width=0.5\textwidth]{img/periodische_orbits/transversal_rueckkehrzeit.pdf}
		\caption{Transversale Menge $\Sigma_{x_p}$, sowie erste R"uckkehrzeit}
\end{figure}
\begin{lemma}\label{rueckkehrzeit_diffbar}
Sei $v \in C^k(\RR^n,\RR^n)$ ein Vektorfeld mit $k \in \mathbb{N}$. Dann existiert eine Umgebung $V\subseteq \RR^n$ von $x_p$, sodass $\tau \in C^k(V, \RR)$.
\end{lemma}
\begin{beweis}
Definiere Funktion $F \colon \RR \times \RR^n \to \RR, \ (t, x) \mapsto \langle \phi(t, x) - x_p, n\rangle$. F ist $k$-fach stetig differenzierbar. Wir weisen die Voraussetzungen f"ur den Satz von der impliziten Funktion nach
\begin{itemize}
	\item Es gilt $F(T, x_p) = 0$ 
	\item Die Ableitung von $F$ nach $t$ ist invertierbar in $x_p$
	\begin{align*}
		&\frac{\mathrm d}{\mathrm dt} \langle \phi(t, x) - x_p, n\rangle =  \langle \frac{\mathrm d}{\mathrm dt} \phi(t, x), n\rangle = \langle v(x), n \rangle\\
		&\Rightarrow \left. \frac{\mathrm d}{\mathrm dt} \langle \phi(t, x) - x_p, n\rangle \right|_{x=x_p} = \langle v(x_p), n \rangle \neq 0
	\end{align*}
\end{itemize}
Der Satz von der impliziten Funktion anwendbar und es existiert daher ein $V\subseteq \RR$, sowie $f \in C^k(V, \RR)$, sodass gilt
\begin{enumerate}
	\item $f(x_p) = T$
	\item $\forall x \in V: F(f(x), x) = 0$
\end{enumerate}
Dieses $f$ stellt die erste R"uckkehrzeit $\tau$ dar, denn es gilt
$$ F(\tau(x), x) = \langle \phi(\tau(x), x) - x_p, n\rangle = 0 = F(f(x), x) $$
\qed
\end{beweis}
\begin{definition}
Die Abbildung
$$P_{\Sigma_{x_p}} \colon V \cap \Sigma_{x_p} \to \Sigma_{x_p},\ x\mapsto \phi(\tau(x), x)$$
hei"st \emph{Poincar\'{e} Abbildung (des periodischen Orbits $\Gamma_{x_p}$ bez"uglich $\Sigma_{x_p}$)}. 
\end{definition}
\begin{description}
\item[Bemerkung] Falls $v\in C^k$, so ist $P_{\Sigma_{x_p}} \in C^k$. Dies ist eine direkte Folgerung von \eqref{rueckkehrzeit_diffbar}, sowie der Eigenschaft, dass $\phi \in C^k$. Die Poincar\'{e} Abbildung besitzt einen Fixpunkt, denn ${P_{\Sigma_{x_p}}(x_p) = x_p}$. Allgemeiner gilt folgendes Lemma 
\end{description}

\begin{lemma}
Sei $x$ ein Fixpunkt von $P_{\Sigma_{x_p}}^N$ mit einem minimalen $N \in \mathbb{N}$. Dann ist $\Gamma_x$ ein periodischer Orbit mit Periode
$$ \sum_{j=1}^N{\tau(x_j)}$$
wobei $x_1 = x, \ x_{j+1} = \phi(\tau(x_j), x_{j})$ f"ur $j = 1,\ldots N$ \
\end{lemma}
\section{Stabilit"atsanalyse periodischer Orbits mittels Poincar\'{e} Abbildung}
\begin{definition}[Orbitale dynamische Stabilit"at]
Sei $(X,d)$ ein metrischer Raum, $(X,\phi)$ ein dynamisches System mit einem periodischen Orbit $\Gamma_{x_p}$. Dann hei"st $\Gamma_{x_p}$ 
\begin{itemize}
\item \emph{orbital stabil}, falls 
	$$ \forall \varepsilon > 0 \exists \delta > 0 \forall x\in X, t\geq 0:
	\operatorname{dist}(x, \Gamma_{x_p}) < \delta \Rightarrow \operatorname{dist}(\phi(t,x), \Gamma_{x_p}) < \varepsilon$$
\item \emph{orbital instabil }, falls $\Gamma_{x_p}$ nicht orbital stabil ist.
\item \emph{orbital asymptotisch stabil}, falls $\Gamma_{x_p}$ orbial stabil ist und gilt
$$ \exists b > 0 \forall x \in X: \operatorname{dist}(x, \Gamma_{x_p}) < b \Rightarrow \lim_{t\to\infty}{\operatorname{dist}(\phi(t,x), \Gamma_{x_p})} = 0 $$
\end{itemize}
wobei $\operatorname{dist}(x, M) := \inf_{y\in M}{d(x,y)}, \ M\subseteq X$. Zu orbital asymptotisch stabilen Orbits $\Gamma_{x_p}$ sagt man auch \emph{Grenzzykel}.
\end{definition}
\begin{figure}[htpb]
		\centering
		\includegraphics[width=0.4\textwidth]{img/periodische_orbits/orbiale_stabilitaet.pdf}
		\includegraphics[width=0.4\textwidth]{img/periodische_orbits/orbitale_asymp_stabilitaet.pdf}
		\caption{Orbitale Stabilit"at(links); Orbitale asymptotische Stabilit"at (rechts)}
\end{figure}
\begin{satz}[Stabilit"atskriterium]\label{stabilitatskriterium_periodische_orbits}
Sei $\Gamma_{x_p}$ ein periodischer Orbit von $(X, \phi)$, $\Sigma_{x_p}$ ein Poincar\'{e} Schnitt durch $x_p$ und $P_{\Sigma_{x_p}}$ eine zugeh"orige Poincar\'{e} Abbildung. Es existiert eine Umgebung $V\subseteq \RR^n$ von $x_p$, sodass $(\Sigma_{x_p}\cap V, \psi)$ ein diskretes dynamisches System durch $\psi(n, x) := P_{\Sigma_{x_p}}^n(x)$ induziert, das den Gleichgewichtspunkt $x_p$ besitzt. 
Dann sind "aquivalent
\begin{enumerate}
\item $x_p$ ist ein (asymptotisch) stabiler Gleichgewichtspunkt des diskreten Systems im Sinne von Lyapunov
\item $\Gamma_{x_p}$ ist ein orbital (asymptotisch) stabiler Orbit des kontinuierlichen Systems.
\end{enumerate}
\end{satz}
\begin{figure}[htpb]
		\centering
		\includegraphics[width=1\textwidth]{img/periodische_orbits/stabilitaetskriterium.pdf}
		\caption{Illustration des Satzes "uber das Stabilit"atskriterium. $\Gamma_{x_p}$ ist orbital asymptotisch stabil. Kontinuierliche System (rechts); Das dazugeh"orige diskretisierte Poincar\'{e} System (links)}
\end{figure}

\begin{beispiel}
Betrachte folgende Differentialgleichung
\begin{align*}
	\begin{cases}
		&\dot x = \mu x - y - x(x^2+y^2)\\
		&\dot y = x + \mu y - y(x^2+y^2)
	\end{cases}
\end{align*}
mit einem Parameter $\mu > 0$. Eine Transformation in Polarkoordinaten vermittels $x = r \cos(\theta),\ y = r \sin(\theta), \ r\geq 0, \ \theta \in [0,2\pi)$ liefert
\begin{align*}
	\begin{cases}
		&\dot r = \mu r - r ^3\\
		& \dot \theta = 1
	\end{cases}
\end{align*}
Daher ist die Flussabbildung folgenderma"sen gegeben
$$\phi\left(t, \begin{pmatrix} r_0\\ \theta_0 \end{pmatrix}\right) = \begin{pmatrix}
	\left(\frac 1\mu + e^{-2\mu t}\left(\frac{1}{r_0^2} - \frac 1\mu\right)\right)^{-\frac 12}\\
	t + \theta_0
\end{pmatrix}$$
Ein periodischer Orbit $\Gamma$ ist offensichtlich gegeben durch 
\begin{align*}
\begin{cases}
 &r = \sqrt\mu\\
 &\theta = \theta_0
\end{cases}
\Leftrightarrow
\begin{cases}
 & x(t) = \sqrt\mu \cos\left(\theta(t)\right)\\
 & y(t) = \sqrt\mu \sin\left(\theta(t)\right)
\end{cases}
\end{align*}
Also hat dieser Orbit die Periode $2\pi$ und er besitzt die Poincar\'{e} Abbildung
$$ P_\Sigma (r_0) = \left(\frac 1\mu + e^{-2\mu t}\left(\frac{1}{r_0^2} - \frac 1\mu\right)\right)^{-\frac 12}$$
wobei $\Sigma = \RR\times\{0\}$, falls $\theta_0 \neq \frac{(2k+1)\pi}{2}$, ansonsten $\Sigma = \{0\}\times\RR$. F"ur alle $\mu > 0$ gilt $P_\Sigma(\sqrt \mu) = \sqrt \mu$. Die Ableitung von $P_\Sigma$ nach $r$ ist 
$$\frac{\mathrm d}{\mathrm dr} P_\Sigma(\sqrt \mu) = e^{-4 \pi \mu} \stackrel{\mu > 0} < 1$$
Die direkte Methode von Lyapunov liefert, dass $(\sqrt \mu, \theta_0)^T$ asymptotisch stabil ist im Sinne von Lyapunov und somit liefert \eqref{stabilitatskriterium_periodische_orbits}, dass $(\sqrt \mu, \theta_0)^T$ orbital asymptotisch stabil ist.
\end{beispiel}

\begin{figure}[htpb]
		\centering
		\includegraphics[width=0.5\textwidth]{img/periodische_orbits/beispiel_asymp_stabilitaet.pdf}
		\caption{Flussabbildung zum Beispiel}
\end{figure}


\section{Poincar\'{e}-Bendixson-Theorie}

Betrachte das GDG-System
	\[\dot{x} = v(x), \qquad x \in\mathbb{R} \]
\begin{definition}
	Sei \(\phi(t,x)\) Flu"sabbildung dieses Systems und \(x_0 \in \mathbb{R}\). Dann hei"st
		\[\omega(x_0)= \{x \in \mathbb{R}^n\ |\ \exists\  (t_j)_{j \in \mathbb{N}} \in \mathbb{R}, t_j \to \infty : \lim_{j\to \infty} \phi(t_j,x) = x_0\} \]
	\(\omega\)-Limesmenge des (Anfangs-) zustands \(x_0\). Jedes \(x \in \omega(x_0)\) ist ein 		sogenannter \emph{\(\omega\)-Limespunkt} von \(x_0\).

\end{definition}
\begin{description}
	\item[Bemerkung] Entsprechend definiert man \(\alpha\)-Limesmengen bzw. \(\alpha\)-Limespunkte im Fall \((t_j) \to -\infty\).
\end{description}

\begin{beispiel}
	\begin{enumerate}
		\item Sei \(x_G\) asymptotisch stabiler Gleichgewichtspunkt. Dann gilt \(\omega(x_0) = \{x_G\} \) f"ur alle \(x_0\) hinreichend nahe bei \(x_G\)
		\item Sei \(\Gamma_{x_p} \) ein orbital asymptotisch stabiler periodischer Orbit. Dann gilt \(\omega(x_0)= \Gamma_{x_p}\) f"ur alle \(x_0\) hinreichend nahe bei \(\Gamma_{x_p}\)
	\end{enumerate}
\end{beispiel}

\begin{definition}
Ein Orbit $\Gamma$ hei"st \emph{heterokliner Orbit} zwischen $x_1$ und $x_2$, falls ein \emph{heterokliner Punkt} $x_h \in \Gamma$ existiert, sodass 
\[\lim_{t\to -\infty} \phi(t,x_h) = x_1,\qquad \lim_{t\to\infty} \phi(t,x_h) = x_2\]
gilt. Seien \(x_G^1,...,x_G^N\) Gleichgewichtspunkte, $x_G^{N+1}$ bezeichne $x_G^1$. Seien $\Gamma_{k}$ heterokline Orbits zwischen $x_G^k$ und $x_G^{k+1}$ mit heteroklinen Punkt $x_h^k$. Dann hei"st die Menge 
	$$	\bigcup_{k=1}^{N} \Gamma_k \cup x_G^k$$
\emph{heterokliner Zykel}.
\end{definition}
\begin{description}
\item[Bemerkung] Es ist in der Definition eines heteroklinen Orbits auch zugelassen, dass dieser Orbit zwischen zwei gleichen Punkten verl"auft, d.h. $x_1 = x_2$. Ein solcher Orbit wird auch als \emph{homokliner Orbit} bezeichnet.
\end{description}
\begin{figure}[htpb]
		\centering
		\includegraphics[width=0.4\textwidth]{img/poincare_Bendixson/heterokline_orbits.pdf}
		\caption{Illustration eines heteroklinen Zykels mit einem homoklinen Orbit zwischen $x_G^4$ und $x_G^5$}
\end{figure}


\begin{satz}[Poincar\'{e}-Bendixson-Theorem]
	Sei \(n=2, M \subset \mathbb{R}^2\) eine positiv invariante, kompakte Teilmenge. Dann gilt f"ur jedes \(x_0 \in M\) hinsichtlich der \(\omega\)-Limesmenge \(\omega(x_0)\) von \(x_0\) eine der folgenden drei Alternativen:
	\begin{enumerate}
		\item \(\omega (x_0) = \{x_G\} \) ist ein Gleichgewichtspunkt in M
		\item \(\omega(x_0) = \Gamma_{x_p}\) ist ein periodischer Orbit
		\item \(\omega(x_0)\) ist ein heterokliner Zykel
	\end{enumerate}

\end{satz}

\begin{corollar}
Es seien die Vorraussetzungen des Poincar\'{e}-Bendixson-Theorems gegeben. Ferner existiere in M kein Gleichgewichtspunkt des Systems. Dann enth"alt M mindestens einen periodischen Oribit des Systems.
\end{corollar}

\begin{beispiel}
	\begin{align*} \begin{cases}
		\dot{x} = \mu x - y - x(x^2+y^2) \\
		\dot{y} = x + \mu y - y(x^2+y^2)
	\end{cases} \end{align*}
	\(x=y=0\) ist trivialer (und einizger) Gleichgewichtspunkt. \\
	Betrachte das Vektorfeld \(v(x,y)\) des Systems l"angs eines Kreises \({x^2+y^2=R^2}\)
		\[\Rightarrow v(x,y) = \begin{pmatrix} \mu x - y - R^2x \\
		x + \mu y - R^2y \end{pmatrix} \]
		\begin{align*}
			\langle v(x,y), \begin{pmatrix} x \\ y \end{pmatrix} \rangle & = \mu x^2 - xy - R^2x^2 + xy + \mu y^2 - R^2y^2\\
			& = (\mu-R^2)(x^2+y^2) = (\mu - R^2)R^2 \lessgtr 0,\qquad ( R \gtrless \sqrt{\mu})
		\end{align*}
	Au"serhalb von \(x=y=0\) exisitiert kein weiterer Gleichgewichtspunkt, da \({\langle v(x,y),\begin{pmatrix} x \\ y \end{pmatrix} \rangle \not= 0}\) f"ur \(R\not= 0, \sqrt{\mu}\) und \(v_{|_{x^2+y^2=\mu}} = \begin{pmatrix} -y \\ x \end{pmatrix} \not= 0\).\\
	Somit existiert nach Poincar\'{e}-Bendixson innerhalb des Ringelements \newline \(R_1^2 < x^2+y^2 < R_2^2\) wenigstens \underline{ein} periodischer Orbit \(\Gamma_{x_p}\).
\end{beispiel}

\begin{figure}[htpb]
		\centering
		\includegraphics[width=0.4\textwidth]{img/poincare_Bendixson/beispiel_poincare_Bendixson.pdf}
		\caption{Vektorfeld auf den Kreisen $R_1$ sowie $R_2$ des obigen Beispiels}
\end{figure}
\section{Zeitlich periodische nicht-autonome GDG-Systeme}

Betrachte die Differentialgleichung
	\begin{align*} 
		&\dot{x} = v(t,x), \qquad x \in \mathbb{R}^n, t \in \mathbb{R}\ \text{mit} \\
		&v(t+T,x) = v(t,x)\qquad \forall (t,x) \in \mathbb{R}\times\mathbb{R}^n \\
	\end{align*}
wobei $T > 0$ minimal ist und die zeitliche Periode des System angibt. Da dies eine nicht-autonome Differentialgleichung ist, wird dadurch a priori kein dynamisches System erzeugt. Doch wenn man den erweiterten Phasenraum betrachtet wird ein dynamisches System induziert.
\begin{lemma}
Jede nicht-autonome Differentialgleichung $\dot x = v(t, x)$ kann folgenderma"sen in eine autonome Differentialgleichung 
transformiert werden
\begin{align*}
	\dot{\tilde x} := \dot{\begin{pmatrix}
		t\\
		x
	\end{pmatrix}} = 
	\begin{pmatrix}
		1\\
		v(t,x)
	\end{pmatrix} = 
	\begin{pmatrix}
		1\\
		v(\tilde x)
	\end{pmatrix}
	=: \tilde v(\tilde x)
\end{align*}
Dabei erweitert man den Phasenraum der nicht-autonomen Differentialgleichung auf $\RR \times \RR^n$, d.h. $\tilde v\colon \RR^{n+1} \to \RR^{n+1}$. Die L"osungen der autonomisierten Differentialgleichung mit Anfangswert $\tilde x (0) = \begin{pmatrix} \tau_0 \\ x_0 \end{pmatrix}$ entspricht der L"osung der nicht-autonomen Differentialgleichung mit Anfangswert $x(\tau_0) = x_0$.
\end{lemma}
Aus dem Lemma folgt sofort, dass die autonomisierte Differentialgleichung auf $\RR\times \RR^n$ ein dynamisches System induziert, falls $v$ entsprechende Bedingungen besitzt. Die Flussabbildung schreiben wir dann folgenderma"sen
\begin{align*} 
		\tilde x = \tilde\phi\left(t, \left(\tau_0, x_0\right) \right)
\end{align*}

Da $v$ in der ersten Komponente $T$-periodisch sind die L"osung mit Anfangswert $(\tau_0, x_0)$ identisch zu den L"osungen mit Anfangswert $(\tau_0 + kT, x_0)$ f"ur jedes $k\in\mathbb{Z}$. Daher ergibt sich auf kanonische Art eine Poincar\'{e}-Abbildung
\[P_{\tau_0} : \mathbb{R}^n \to \mathbb{R}^n, \ x_0\mapsto \phi(T;\tau_0,x_0) \]
wobei $\phi$ der Fluss der nicht-autonomen Differentialgleichung ist. Ein dazugeh"origer Poincar\'{e} Schnitt ist beispielsweise
$$ \Sigma_{\tau_0} = \left \{ \left. (t,x)\in\RR\times\RR^n \right | \langle (t - \tau_0 ,x), (1,0)\rangle = 0 \right \} = \left \{ \left . (\tau_0, x) \right| x \in \RR^n \right \} $$
Im folgenden werden wir diese spezielle Poincar\'{e} Abbildung auch mit \emph{Periodenabbildung} bezeichnen.
\begin{figure}[htpb]
		\centering
		\includegraphics[width=0.7\textwidth]{img/poincare_Bendixson/periodenabbildung.pdf}
		\caption{Illustration der Periodenabbildung; $\Gamma_{x_1}$ ist $T$-periodisch, $\Gamma_{x_2}$ ist $2T$-periodisch}
\end{figure}


Die Periodenabbildung erzeugt analog wie im vorherigen Kapitel ein diskretes dynamisches System durch $\psi(k, x) = P_{\tau_0}^k(x)$. Daher kann man die periodischen Orbits von $\phi$ wieder mithilfe der Stabilit"at von Gleichgewichtspunkten von $\psi$ analysieren.

\begin{description}
	\item[Bemerkung] Fixpunkte von \(P_{\tau_0}\) entsprechen i.A. einem T-periodischen Orbit von \(\dot{x} = v(t,x) \) und Fixpunkte von \(P_{\tau_0}^K\) (\(K\in\mathbb{N}\)) entsprechen einem KT-periodischen Orbit einschlie"slich der Stabilt"atseigenschaften.
\end{description}

\begin{beispiel}
	\[\dot{x} = -x + \sin t \qquad (\text{nicht autom,} 2\pi \text{-periodisch})\]
	Die allgemeine homogene L"osung ist gegeben durch \(x_h(t) = e^{(t-t_0)}x_0\)
	Allgemeine L"osung: 
		\begin{align*} x(t) &= x_p(t) + x_h(t) \\
			& = \frac 1 2 (\sin t - \cos t) - \frac 1 2 e^{(t-t_0)}(\sin t_0 - \cos t_0) + e^{(t-t_0)}x_0\\
			&= \phi(t; t_0,x_0)
		\end{align*}
	\(\Rightarrow\) Mit \(\tau_0=t_0=0, t = 2\pi\) folgt:
		\begin{align*} P_0 : \mathbb{R} \to \mathbb{R}, x_0 \mapsto &-\frac 1 2 - \frac 1 2 e^{-2\pi}(-1)+e^{-2\pi}x_0\\
			& = -\frac 1 2 + \frac 1 2 e^{-2\pi} + e^{-2\pi}x_0
		\end{align*}
	ist Periodenabbildung f"ur obige GDG.\\
	Bestimmung des (eindeutigen) Fixpunkts: 
		\begin{align*} P_0(x_0)=x_0 &\Leftrightarrow -\frac 1 2 + \frac 1 2 e^{-2\pi} + e^{-2\pi}x_0 = x_0 \\
		&\Rightarrow x_0 = \frac{-\frac 1 2 + \frac 1 2 e^{-2\pi}} {e^{-2\pi}}
		\end{align*}
		\(\Rightarrow 0 < \frac{d}{d_x0} P_0(x_0) = e^{-2\pi}< 1 \Rightarrow\) asymptotisch stabil \(\Rightarrow\) obige GDG besitzt einen orbital asymptotisch stabilen \(2\pi\)-periodischen Orbit.
	
\end{beispiel}

%-----------------------------------------------------------------------------------------------

\chapter{Verzweigungstheorie (Bifurkationstheorie)}

Zun"achst werden \emph{station"are Verzweigungen} betrachtet.

\section{Kontinuierlicher Fall f"ur $n=1$}

F"ur $(\lambda,x)\in\RR\times\RR$ und den \emph{Verzweigungsparameter} $\lambda$ betrachte man
\[
\dot x = v(\lambda,x)
\]

Bei der \emph{station"aren Verzweigungstheorie} studiert man die Struktur der Gleichgewichtspunkte im Phasenraum ($x$-Raum) in Abh"angigkeit vom Parameter $\lambda\in\RR$.
Im Folgenden sei o.B.d.A. $x=0$ f"ur alle Werte von $\lambda$ ein trivialer Gleichgewichtspunkt. Das hei"st es gilt f"ur alle $\lambda \in \RR: v(\lambda, 0) = 0$
\begin{definition}
Die Menge aller trivialen Gleichgewichtspunkte bildet den \emph{Grundl"osungszweig}
$$ G= \left \{ (\lambda, 0) \in \RR \times \RR \right \}$$
\end{definition}

Falls der Grundl"osungszweig die Form $x=x_G(\lambda)$ mit $\lambda\in\RR$ hat, setze
\begin{align*}
x &=x_G(\lambda)+\xi\\
\Rightarrow\dot x &= \dot\xi = v(\lambda,x_G(\lambda)+\xi) = \tilde v(\lambda,\xi).
\end{align*}
Da $v(\lambda, x_G(\lambda)) = 0$, ist $\xi = 0$ Gleichgewichtspunkt f"ur alle $\lambda$

\begin{definition}
Ein Punkt $(\lambda_C,0) \in G$ auf dem Grundl"osungszweig hei"st \emph{station"arer Verzweigungspunkt (Bifurkationspunkt)} des Problems $\dot x = v(\lambda, x)$, falls er in $\RR\times\RR$ H"aufungspunkt nicht-trivialer Gleichgewichtsl"osungen $(\lambda,x_G)$ mit ${x_G\neq 0}$ ist.
\end{definition}
Im Folgenden bezeichnet $v_x$ die partielle Ableitung von $v$ nach $x$
$$ v_x = \partial_x v = \frac{\partial}{\partial x} v$$
\begin{lemma}
Sei $U\subseteq \RR^2$ offen, sowie $v \in C^1(U, \RR)$. Eine notwendige Bedingung f"ur einen  Verzweigungspunkt $(\lambda_C, 0)\in G$ ist
\[
v_x(\lambda_C,0) = 0
\]
\end{lemma}

\begin{beweis}
Angenommen $v_x(\lambda_C,0)\neq 0$. Dann folgt nach dem Satz "uber implizite Funktionen f"ur \(v(\lambda_C,0)=0, v_x(\lambda_C,0) \not= 0\), dass \(v(\lambda,x)=0\) nahe \((\lambda_C,0)\) zu jedem \(\lambda\) genau einen Gleichgewichtspunkt \(x=x_G(\lambda)\) hat, mit \(x_G(\lambda)\ C^1\)-glatt, \(x(\lambda_C)=0\).
 Damit gilt notwendigerweise $x_G(\lambda)\equiv 0$, d.h.\ nahe $(\lambda_C,0)$ existiert keine nicht-trivialen L"osungspunkte.
 \qed
\end{beweis}

\begin{definition}
Ein Verzweigungspunkt $(\lambda_C, x_C)$ hei"st \emph{transkritisch}, falls in jeder hinreichend kleinen Umgebung $U$ von $(\lambda_C, x_C)$ Parameter $\lambda_- < \lambda_C < \lambda_+$ und Anfangswerte $x_+, x_- \in \RR$ existieren, sodass $(x_- - x)(x_+ - x) < 0$ und 
$$ v(\lambda_-, x_-) = v(\lambda_C, x) = v(\lambda_+, x_+) = 0$$
\end{definition}
\begin{figure}[htpb]
		\centering
		\includegraphics[width=0.4\textwidth]{img/bifurkation/transkritischer_vp.pdf}
		\includegraphics[width=0.4\textwidth]{img/bifurkation/nicht_transkritischer_vp.pdf}
		\caption{Transkritischer Verzweigungspunkt (links); Kein transkritischer Punkt (rechts)}
\end{figure}

\begin{definition}
Ein Verzweigunspunkt $(\lambda_C, x_C)$ hei"st \emph{subkritisch} bzw. \emph{superkritisch}, falls eine Umgebung $U$ von $(\lambda_C, x_C)$ existiert, sodass f"ur alle nicht-trivialen Gleichgewichtspunkte ($\lambda, x) \in U$ gilt
$$ \lambda < \lambda_C \mbox{ bzw. } \lambda > \lambda_C$$
\end{definition}

\begin{figure}[htpb]
		\centering
		\includegraphics[width=0.4\textwidth]{img/bifurkation/subkritischer_vp.pdf}
		\includegraphics[width=0.4\textwidth]{img/bifurkation/superkritischer_vp.pdf}
		\caption{Subkritischer Verzweigungspunkt (links); Superkritischer Verzweigungspunkt (rechts)}
\end{figure}

\begin{satz}[Hinreichende Bedingung f"ur einen VP]\label{hinreichend_verzweigungspunkt}
Sei $U\subseteq \RR^2$ offen, $v \in C^k(U, \RR)$ f"ur ein $k\geq 2$ und $v(\lambda_C,0)=0$. Es gelten weiter 
\begin{enumerate}
\item\label{enum:1}
$v_x(\lambda_C,0) = 0$

\item\label{enum:2}
$v_{\lambda x}(\lambda_C,0)\neq 0$.
\end{enumerate}
Dann ist $(\lambda_C,0)$ ein Verzweigungspunkt. Weiterhin existiert in einer Umgebung von $0$ ein eindeutiger nicht-trivialer L"osungszweig $\lambda = \lambda^*(x) \in C^{k-1}(\RR)$, welcher den Grundl"osungszweig in $(\lambda_C,0)$ \emph{transversal} schneidet in $(\lambda_C,0)$, d.h.\ $\lambda^*(0) = \lambda_C$ und 
$$-\frac{v_{xx}(\lambda_C,0)}{2v_{\lambda x}(\lambda_C,0)} = {\lambda^*}'(0)  \in \RR$$

Gilt zudem
\begin{enumerate}
\setcounter{enumi}{2}
\item\label{enum:3}
$v_{xx}(\lambda_C,0)\neq 0$.
\end{enumerate}
Dann ist die Verzweigung bei $(\lambda_C,0)$ transkritisch.

Falls anstelle von \ref{enum:3}
\begin{enumerate}
\setcounter{enumi}{3}
\item\label{enum:4}
\(
v_{xx}(\lambda_C,0) = 0\\
v_{xxx}(\lambda_C,0)\neq 0
\)
\end{enumerate}
mit $k\geq3$ gilt, dann ist die Verzweigung \emph{super- } bzw.\ \emph{subkritisch} falls $$-\frac{v_{xxx}(\lambda_C, 0)}{3v_{\lambda x}(\lambda_C, 0)} = {\lambda^*}''(0)$$ positives bzw. negatives Vorzeichen hat.
\end{satz}

\begin{beweis}
Man betrachte die \emph{Gleichgewichtsbedingung (station"ar)}
\[
v(\lambda,x) = 0.
\]
und setze 
\[
V(\lambda,x) =
\begin{cases}
\frac{v(\lambda,x)}{x} & x\neq 0\\
v_x(\lambda,0) & x = 0
\end{cases}
\]
mit $(\lambda,x)\in\RR\times\RR$ beliebig, woraus die $C^{k-1}$-Glattheit von $V$ folgt.
\\
Aus \ref{enum:1} folgt $V(\lambda_C,0) = 0$. Ferner gilt $V_\lambda(\lambda_C,0)\neq 0$, weshalb sich $V(\lambda,x) = 0$ lokal eindeutig nach  $\lambda = \lambda^*(x)$ aufl"osen l"asst (Satz "uber implizite Fuktionen) mit $\lambda^*(0) = \lambda_C$, $\lambda^*$ $C^{k-1}$-glatt.
\\
Insbesondere gilt: $v(\lambda^*(x),x) = 0$ f"ur alle $x\neq 0$
\\
\underline{Taylorentwicklung} von $v(\lambda,x)$ um $(\lambda_C,0)$:
\begin{align*}
v(\lambda,x)=&\underbrace{a(\lambda)}_{v_x(\lambda,0)}x + \underbrace{b(\lambda)}_{\frac{1}{2}v_{xx}(\lambda,0)}x^2 + \underbrace{c(\lambda)}_{\frac{1}{6}v_{xxx}(\lambda,0)}x^3 + \dots\text{, falls } v \text{ entsprechend glatt}\\
=& v_{\lambda x}(\lambda_C,0)(\lambda-\lambda_C)x +\dots\\
&+ \frac{1}{2}v_{xx}(\lambda_C,0)x^2 + \dots\\
&+ \frac{1}{6}v_{xxx}(\lambda_C,0)x^3 + \dots
\end{align*}
Daraus folgt:
\begin{align*}
V(\lambda,x) =&  v_{\lambda x}(\lambda_C,0)(\lambda - \lambda_C) + \dots \\
&+ \frac{1}{2}v_{xx}(\lambda_C,0)x + \dots\\
&+\frac{1}{6}v_{xxx}(\lambda_C,0)x^2 + \dots
\end{align*}
Daher gilt:
\begin{align*}
V_\lambda(\lambda_C,0) &= v_{\lambda x}(\lambda_C,0)\neq 0\\
V_x(\lambda_C,0) &= \frac{1}{2}v_{xx}(\lambda,0)\\
V_{xx}(\lambda_C,0) &= \frac{1}{3}v_{xxx}(\lambda,0)
\end{align*}
\\\underline{Zusatzaussage:}
\\Aus \ref{enum:3} folgt f"ur hinreichend kleine $|x|$
$$ V(\lambda^*(x),x) = 0 $$
Differenzieren der impliziten Darstellung ergibt 
$$V_x(\lambda^*(x),x) + V_\lambda(\lambda^*(x),x)\cdot(\lambda^*)'(x) = 0$$
Speziell f"ur $x=0$ folgt, dass
\[
V_x(\lambda_C,0) + \underbrace{V_\lambda(\lambda_C,0)}_{\neq 0\text{ wegen \ref{enum:2}}}\cdot(\lambda^*)'(0) = 0
\]
und somit
\[
(\lambda^*)'(0) = -\frac{V_x(\lambda_C,0)}{V_\lambda(\lambda_C,0)} = -\frac{v_{xx}(\lambda_C,0)}{2v_{\lambda x}(\lambda_C,0)}\stackrel{\ref{enum:3}}{\neq}0
\]
Daher ist $(\lambda_C,0)$ ein transkritischer Verzweigungspunkt.
\\
Falls \ref{enum:4} gilt, so kann ein weiteres mal differenziert werden und es gilt
\begin{align*}
V_{x\lambda}(\lambda^*(x),x)\cdot(\lambda^*)'(x) + V_{xx}(\lambda^*(x),x) + V_{\lambda\lambda}(\lambda^*(x),x)\cdot(\lambda^*)'(x)^2\\
 + V_{\lambda x}(\lambda^*(x),x)\cdot(\lambda^*)'(x) + V_\lambda(\lambda^*(x),x)(\lambda^*)''(x) = 0
\end{align*}
Speziell f"ur $x = 0$ gilt $\lambda^*(x) = \lambda_C$, sowie $(\lambda^*)'(x) = 0$ und es fallen alle Terme mit $(\lambda^*)'(x)$ weg. Daher ergibt sich
\begin{align*}
V_{xx}(\lambda_C,0) + V_\lambda(\lambda_C,0)\cdot(\lambda^*)''(0)=0
\end{align*}
Letztendlich folgt daraus, dass
\begin{align*}
(\lambda^*)''(0) = -\frac{V_{xx}(\lambda_C,0)}{V_\lambda(\lambda_C,0)} = -\frac{v_{xxx}(\lambda_C,0)}{3v_{\lambda x}(\lambda_C,0)}\neq 0
\end{align*}
Ist nun $(\lambda^*)''(0) > 0$, so handelt es sich um ein superkritischen Verzweigungspunkt, f"ur $(\lambda^*)''(0) < 0$ liegt ein subkritischer Verzweigungspunkt vor.
\qed
\end{beweis}

\begin{beispiel}
$\dot x = \lambda x - x^2$, $v(\lambda,0) = 0$ f"ur alle $\lambda$, $\underbrace{v_x(\lambda,0)}_{= (\left.\lambda - 2x)\right|_{x = 0}}\stackrel{!}{=}0$.

$\Rightarrow\lambda = \lambda_C = 0$  ist kritischer Punkt, Bedingung \ref{enum:1} ist erf"ullt (sonst nirgends VP).
\begin{align*}
v_{x\lambda}(\lambda_C,0) &= 1\neq 0\text{, daher ist } (\lambda_C,0)\text{ VP,}\\
v_{xx}(\lambda,0) &= -2,\text{inbesondere},v_{xx}(0,0) = -2\neq 0 \\
\Rightarrow(\lambda_C,0) &= (0,0)\text{ ist transkritischer VP.}
\end{align*}
\end{beispiel}

\section{Diskreter Fall f"ur $n=1$}
Sei $\psi\colon\RR\times\RR\to\RR$, sodass $\psi(\lambda,\cdot)\colon\RR\to\RR$ ein Hom"oomorphismus bzw. $C^k$-Diffeomorphismus f"ur alle $\lambda \in \RR$ ist.
\\
$\psi(\lambda,0) = 0$ f"ur alle $\lambda\in\RR$, d.h. $x = 0$ ist f"ur alle $\lambda\in\RR$ ein trivialer Fixpunkt.

Nicht-triviale Gleichgewichtspunkte: 
\begin{align*}
&\psi(\lambda,x)=x\\
\Leftrightarrow &\underbrace{\psi(\lambda,x)-x}_{= v(\lambda,x)} = 0
\end{align*}
Das zugeh"orige station"are Problem ist formal identisch mit jenem des kontinuierlichen Falls weshalb der Satz entsprechend anwendbar ist.


%% -- VL 09.12.13

\section{Stabilit"atsanalyse}
Wir wollen nun die Stabilit"atseigenschaften der Gleichgewichtspunkte entlang von Verzweigungen analysieren. Die Theorie von Lyapunov "uber Gleichgewichtspunkte, insbesondere die direkte Methode von Lyapunov, wird dabei sehr hilfreich sein. Um diese Anwenden zu k"onnen, ben"otigen wir die Eigenwerte der Jacobi-Matrix von $v$ bei einem nicht-trivialen Gleichgewichtspunkt.
\subsection{Stabilit"atsanalyse im eindimensionalem Fall}
Um Sicherzustellen, dass tats"achlich nicht-triviale Gleichgewichtspunkte existieren, fordern wir die Voraussetzungen vom Satz "uber die hinreichenden Bedingungen f"ur einen Verzweigungspunkt \eqref{hinreichend_verzweigungspunkt}.

Die Analyse im eindimensionalem Fall ist sehr einfach, denn es werden nur Kenntnisse des einzigen Eigenwertes $v_x(\lambda^*(x), x)$ ben"otigt. 

\begin{definition}
Sei $(\lambda_C,0)$ ein Verzweigungspunkt, $\lambda^*$ der zugeh"orige nicht-triviale L"osungszweig. Ein Punkt $(\lambda^*(x), x)$ hei"st \emph{Umkehrpunkt}, falls ${\lambda^*}'(x) = 0$.
\end{definition}

\begin{satz}
Sei $(\lambda_C, 0)$ ein Verzweigungspunkt sowie $v_{\lambda x}(\lambda_C, 0) > 0$. Sei $\lambda^*$ der zugeh"orige nicht-triviale L"osungszweig.
Dann ist der Gleichgewichtspunkt $(\lambda^*(x), x)$ 
\begin{itemize}
\item asymptotisch stabil, falls $x{\lambda^*}'(x) > 0$
\item instabil, falls $x{\lambda^*}'(x) < 0$
\end{itemize}
In einer hinreichend kleinen Umgebung von $(\lambda_C, 0)$, kann das Vorzeichen von ${\lambda^*}'(x)$ durch $v_{xx}(\lambda_C,0)$ ermittelt werden.
\end{satz}

\begin{beweis}

Sei im folgendem $\mu(\lambda):=v_x(\lambda,0)$ die Jakobi-Matrix entlang des Grundl"osungszweiges. Es gilt $v_{\lambda x}(\lambda_C,0) = \mu'(\lambda_C) >0$. Sei $\lambda^*(x)$ der nicht-trivialer L"osungszweig, dieser existiert nach \eqref{hinreichend_verzweigungspunkt} in einer Umgebung $U$ von $0$. Setze nun $\gamma(x):=v_x(\lambda^*(x),x)$ f"ur $x \in U$.

F"ur die Vorzeichenbestimmung von $\gamma(x)$ werden nun die Jakobi-Matrix bzw.\ die Eigenwerte dieser entlang des nicht-trivialen L"osungszweiges untersucht.

Wegen $v(\lambda^*(x),x)=0$ liefert Differentiation nach $x$ f"ur $x\in U$:
\begin{align*}
v_\lambda(\lambda^*(x),x)\cdot(\lambda^*)'(x) + \underbrace{v_x(\lambda^*(x),x)}_{=\gamma(x)} = 0\\
\Rightarrow\gamma(x) = -v_\lambda(\lambda^*(x),x)\cdot(\lambda^*)'(x)
\end{align*}
Taylorentwicklung um $x=0$ liefert nun f"ur $|x|\to 0$:
\begin{align*}
v_\lambda(\lambda^*(x),x) &= \underbrace{v_\lambda(\lambda_C,0)}_{= 0} + (\underbrace{v_{\lambda\lambda}(\overbrace{\lambda^*(0)}^{= \lambda_C},0)}_{= 0}\cdot(\lambda^*)'(0) + v_{x\lambda}(\overbrace{\lambda^*(0)}^{= \lambda_C},0))\cdot x + o(x)\\
\Rightarrow\gamma(x) &= -\underbrace{v_{x\lambda}(\lambda_C,0)}_{= \mu'(\lambda_C)}\cdot x(\lambda^*)'(x) + \underbrace{o(x)(\lambda^*)'(x)}_{= \mu'(\lambda_C)x(\lambda^*)'(x)o(1)}\\
&= \underbrace{(-1+o(1))}_{<0}\underbrace{\mu'(\lambda_C)}_{>0}x(\lambda^*)'(x)
\end{align*}
Daraus ergibt sich nun, falls $(\lambda*(x), x)$ kein Umkehrpunkt ist, dass
\begin{itemize}
\item
f"ur $x<0$ haben $\gamma(x)$ und $(\lambda^*)'(x)$ das gleiche Vorzeichen,

\item
f"ur $x>0$ haben $\gamma(x)$ und $(\lambda^*)'(x)$ entgegengesetztes Vorzeichen
\end{itemize}
Die Gleichgewichtspunkte entlang des nicht-trivialen L"osungszweiges sind somit asymptotisch stabil, wo sich jener f"ur $x>0$ nach rechts wendet ($(\lambda^*)'>0$) und instabil, wo sich jener nach links wendet ($(\lambda^*)'(x)<0$), umgekehrt für \(x<0\).
\qed
\end{beweis}
\begin{figure}[htpb]
		\centering
		\includegraphics[width=0.6\textwidth]{img/bifurkation/stab/stab_analyse.pdf}
		\caption{Stabilit"atsanalyse entlang des L"osungszweigs $\lambda^*$, $x_U^i$ sind Umkehrpunkte}
\end{figure}
\begin{bemerkung}
F"ur einen transkritischen Verzweigungspunkt ergibt obiger Satz, dass der L"osungszweig sowohl stabile, als auch instabile Gleichgewichtspunkte enth"alt, welche durch den Punkt $(\lambda_C, 0)$ getrennt werden. Dies bezeichnet man auch als  \emph{Prinzip des Stabilit"atsaustausches}.
\end{bemerkung}


\begin{beispiel}
\begin{itemize}
\item
\begin{align*}
v(\lambda,x) &= \lambda x-x^2\quad(\lambda\in\RR,x\in\RR)\\
v(\lambda,0) &= 0\text{ f"ur alle }\lambda\\
\mu(\lambda) = v_x(\lambda,0) &= \lambda \stackrel{\lambda_C = 0}{\Rightarrow}\mu(0) = v_{x}(0,0) = 0\\
\mu'(\lambda_C) = v_{x\lambda}(\lambda_C,0) &=1 > 0
\end{align*}
Daher ist $(0,0)$ ein station"arer VP.
\begin{align*}
v_{xx}(\lambda_C,0) &= -2 < 0\\
v_x(\lambda,x) &= \lambda-2x\\
v_{xx}(\lambda,x) &= -2
\end{align*}
Somit handelt es sich um eine \emph{transkritische Verzweigung}.

\item
\begin{align*}
v(\lambda,x) &= \lambda x-x^3\quad(\lambda\in\RR,x\in\RR)\\
v(\lambda,0) &= 0\text{ f"ur alle }\lambda\\
\mu(\lambda) = v_x(\lambda,0) &= \lambda \stackrel{\lambda_C = 0}{\Rightarrow}\mu(0) = v_{xx}(0,0) = 0\\
\mu'(\lambda_C) = v_{x\lambda}(\lambda_C,0) &=1 > 0
\end{align*}
Daher ist $(0,0)$ ein station"arer VP mit kritischem Parameterwert $\lambda_C = 0$.
\begin{align*}
v_x(\lambda,x) &= \lambda - 3x^2\\
v_{xx}(\lambda,x) &= -6x\Rightarrow v_{xx}(\lambda_C,0) = 0\\
v_{xxx}(\lambda,x) &= -6 < 0
\end{align*}
Somit handelt es sich um eine \emph{superkritische Heugabelverzweigung} bei $(0,0)$.
\end{itemize}
\end{beispiel}

\subsection{Stabilit"atsanalyse in beliebiger Dimension}
Man betrachte nun die Differentialgleichung 
$$\dot x = v(x)$$
mit einem Vektorfeld $v\in C^k(\RR \times \RR^n, \RR)$, $k\geq 2$. Es bezeichne im Folgendem $\operatorname{J}_xv$ die \emph{Jacobi-Matrix} von v. Da wir uns in endlich-dimensionalen Vektorr"aumen befinden, werden wir gelegentlich die Jacobi-Matrix als \emph{totale Ableitung} $\operatorname{D}_xv$ schreiben.

\begin{satz}[Station"are \emph{Verzweigung in einem einfachen Eigenwert 0} bzw.\ \emph{station"are \glqq Kodimension 1\grqq-Verzweigung}]
Es habe $\operatorname{J}_xv(\lambda,0)$ einen algebraisch und geometrisch einfachen Eigenwert $\mu(\lambda)$ mit $\mu(\lambda_C) = 0$, $\mu'(\lambda_C) > 0$ f"ur ein $\lambda_C \in \RR$. Die "ubrigen Eigenwerte von $\operatorname{J}_xv(\lambda,0)$ haben einen Realteil ungleich 0. Dann ist $(\lambda_C, 0)$ ein station"arer Verzweigungspunkt.

Falls die "ubrigen Eigenwerte von $\operatorname{J}x_v(\lambda,0)$ f"ur alle $\lambda\in\RR$ einen Realteil strikt kleiner als $0$ besitzen, so gilt das \emph{Prinzip des Stabilit"atsaustausches} wie im Falle $n = 1$.

Sei ferner $\varphi(\lambda)$ ein Eigenvektor zum Eigenwert $\mu(\lambda)$ von $\operatorname{J}_xv(\lambda,0)$ und $\tilde{\varphi} = \tilde{\varphi}(\lambda_C)$ ein Eigenvektor von $\left[\operatorname{J}_xv(\lambda_C,0)\right]^T$ zum Eigenwert $\mu(\lambda_C) = 0$, $\left[\operatorname{J}_xv(\lambda_C,0)\right]^T\tilde{\varphi} = 0$ mit $\langle\tilde{\varphi},\varphi(\lambda_C)\rangle > 0$. \\
Dann folgt
\begin{enumerate}
\item falls
$\langle\tilde{\varphi},D_{xx}^2v(\lambda_C,0)(\varphi(\lambda_C),\varphi(\lambda_C))\rangle\neq 0$, so ist $(\lambda_C,0)$ ein transkritischer Verzweigungspunkt

\item falls $k\geq 3$, 
$\langle\tilde{\varphi},D_{xx}^2v(\lambda_C,0)(\varphi(\lambda_C),\varphi(\lambda_C))\rangle = 0$ und\\ $\langle\tilde{\varphi},D_{xxx}^3v(\lambda_C,0)(\varphi(\lambda_C),\varphi(\lambda_C),(\varphi(\lambda_C))\rangle\stackrel{(<)}{>}0$, so ist $(\lambda_C,0)$ eine subkritische (im Falle von \glqq$>$\grqq) bzw. ein superkitische (im Falle von \glqq$<$\grqq) Heugabelverzweigung.

\end{enumerate}
\end{satz}

\begin{bemerkung}
Falls $\mu(x)$ im vorigen Satz nicht einfacher EW von $J_xv(\lambda,0)$ ist, gilt der Satz im allgemeinen \emph{nicht}.
\end{bemerkung}

\begin{beispiel}
F"ur $\lambda,x,y\in\RR$ sei:
\[
v(\lambda,x,y):=\begin{pmatrix}
\lambda x + y(x^2+y^2)\\
\lambda y - x(x^2+y^2)
\end{pmatrix}.
\]
Dann gilt:
\begin{align*}
v(\lambda,0,0) &= \begin{pmatrix}
0\\
0
\end{pmatrix}\\
\operatorname{J}_xv(\lambda,0,0) &= \left(\begin{array}{cc}
\lambda & 0\\
0 & \lambda
\end{array}\right)
\end{align*}
Daher $\mu(\lambda) = \lambda$ algebraischer und geometrischer EW. Weiter gilt:
\begin{align*}
\mu(0) &= 0\\
\mu'(0) &= 1 > 0\qquad (\lambda_C = 0)
\end{align*}
Aber au"ser diesen trivialen L"osungen gibt es keine weiteren Gleichgewichtspunkte. Dazu setze man $v(\lambda,x,y) = 0$:
\begin{itemize}
\item
f"ur $\lambda = 0$ erh"alt man aus $x^2+y^2 = 0$, dass $x = y = 0$ und die L"osung somit ein Spezialfall der oben betrachteten Gleichgewichtspunkte ist,

\item
f"ur $\lambda\neq 0$ folgt aus $x = 0$ oder $y = 0$, dass wegen $x^2+y^2 = 0$ auch $x = y = 0$ gilt. Seien deshalb nun $x\neq 0$ und $y\neq 0$. L"ost man nun $v_2(\lambda,x,y)=0$ nach $\lambda$ auf erh"alt man:
\[\lambda = \frac{x}{y}(x^2+y^2)\]
Setzt man diese $\lambda$ nun in $v_1(\lambda,x,y)=0$ ein ergibt sich:
\[
\frac{x^2}{y} = -y \Leftrightarrow x^2 = -y^2 \Leftrightarrow x = y = 0
\]
Das steht jedoch im Widerspruch zur Annahme.
\end{itemize}
\end{beispiel}

% 16.12.

\section{Stabilit"atsanalyse f"ur Hom-Systeme}

Sei $\Psi(\lambda,\cdot)\colon\mathbb{R}^n\to\mathbb{R}$ ein $C^2$-Diffeomorphismus f"ur jedes $\lambda\in\mathbb{R}$. 
Sei o.B.d.A $\Psi(\lambda,0)=0$ f"ur alle $\lambda$ in einer Umgebung $U$ von $\lambda_C\in\RR$, d.h. $0$ ist ein Fixpunkt von $\Psi$ und somit ein Gleichgewichtspunkt des von $\Psi$ erzeugten dynamischen Systems. Dies bildet uns wieder einen Grundl"osungszweig.

Wir wollen nun die Theorie im kontinuierlichem Fall auf den diskreten Fall "ubertragen. Dazu stellen wir das Fixpunktproblem, als Nullstellenproblem dar und erhalten dadurch ein Vektorfeld f"ur ein kontinuierliches System
\[\Psi(\lambda,x)=x \ \Leftrightarrow \underbrace{\Psi(\lambda,x)-x}_{=:v(\lambda,x)}=0\]
Daher sind L"osungen des station"aren Verzweigungsproblems im diskreten Fall durch Anwendung der Theorie f"ur den kontinuierlichen Fall auf 
\[v(\lambda,x):=\Psi(\lambda,x)-x\]
"aquivalent.

\begin{satz}
Sei $\Psi\colon\RR\times\RR^n\to\RR^n$ ein $C^2$-Diffeomorphismus. Sei $G\subseteq \RR\times\RR^n$ ein Grundl"osungszweig, $(\lambda_C, x_t)\in G$. Es habe $\operatorname{J}_x \Psi(\lambda, x_t)$ einen algebraisch und geometrisch einfachen Eigenwert $\gamma(\lambda)$ f"ur alle $(\lambda, x_t)$ in einer Umgebung von $(\lambda_C, x_t)$ mit
$$ \gamma(\lambda_C) = 1 , \  \gamma'(\lambda_C) \neq 0$$
Dann ist $(\lambda_C, x_t)$ ein station"arer Verzweigungspunkt des von $\Psi$ erzeugten dynamischen Systems.
\end{satz}

\begin{satz}
Es gelten die Voraussetzungen des vorherigen Satzes. Es sei weiterhin $\gamma'(\lambda_C) > 0$. F"ur alle \emph{anderen} Eigenwerte $\mu \in \sigma(\operatorname{J}_x \Psi(\lambda, x_t))$ gelte $|\mu|< 1$ f"ur alle $(\lambda, x_t)$ in der Umgebung um $(\lambda_C, x_t)$.\\
Dann ist der Verzweigungspunkt $(\lambda_C, x_t)$ transkritisch und es gilt das \emph{Prinzip des Stabilit"atsaustausches}.
\end{satz}

Ebenso gelten die Kriterien f"ur die Verzweigungsrichtung im Verzweigunspunkt $(\lambda_c,0)$ w"ortlich wie im kontinuierlichen Fall , da der modifizierte Term $-x$ keinen Beitrag zu den Ableitungen der Ordnung $\geq 2$ liefert.

\begin{beispiel}
Sei $\lambda\in\mathbb{R}$
\[\Psi(\lambda,x) \begin{cases}
x_1\mapsto x_1+\lambda x_1+x_1^2+x_2^3\\
x_2\mapsto \frac 1 2 x_2+x_1^2+x_2^2
\end{cases}\]
$\Psi(\lambda,x)$ ist $C^{\infty}$-glatt\\
\[\Psi(\lambda,0)=0\ \forall\lambda\in\mathbb{R}\]
$\Rightarrow$ Grundl"osungszweig trivialer (Fix-) Gleichgewichtspunkte\\
\[\operatorname{J}_x\Psi(\lambda,x)=\begin{pmatrix}
1+\lambda+2x_1 & 3x_2^2\\
2x_1 & \frac 1 2 +2x_2\\
\end{pmatrix}\]
\[\operatorname{J}_x\Psi(\lambda,0)=\begin{pmatrix}
1+\lambda & 0\\
0 & \frac 1 2\\
\end{pmatrix}\]
Eigenwerte $\gamma(\lambda)=1+\lambda$
\[\Rightarrow \gamma(0)=1, \gamma'(\lambda)=1>0\ \forall \lambda\in\mathbb{R}\]
\[\mu=\frac 1 2\ \forall \lambda\in\mathbb{R} \Rightarrow |\mu|<1\]
$\Rightarrow (\lambda_c,0)$ ist ein station"arer Verzweigungspunkt $(\lambda_c =0, x=0)$ und es gilt das Prinzip des Stabilitätsaustausches\\
Der Eigenvektor zu $\gamma(\lambda)=1+\lambda$ ist $\varphi(\lambda)=(1,0)\ \forall \lambda\in\mathbb{R}$\\
insbesondere $\varphi =\varphi(\lambda_c)=\varphi(0)=(1,0)$\\
Ferner gilt: $\operatorname{J}_x\Psi(\lambda,0)^{\operatorname{T}}=\operatorname{J}_x\Psi(\lambda,0)$ symmetrische Matrix\\
$\Rightarrow \tilde{\varphi}=\varphi$ ist Eigenvektor zum Eigenwert $1$ von $\operatorname{J}_x\Psi(\lambda,0)^T$\\
$\Rightarrow \langle\tilde{\varphi},\varphi\rangle=\langle\varphi,\varphi\rangle =1>0$
\[\operatorname{D}_{\varphi}\Psi(\lambda,x)=\operatorname{J}_x\Psi(\lambda,x)\varphi=\begin{pmatrix}1+\lambda+2x_1\\ 2x_1 \end{pmatrix}\]
\[\operatorname{D}_{\varphi}(\operatorname{D}_{\varphi}\Psi(\lambda,x))=\begin{pmatrix}2 & 0\\ 2& 0\end{pmatrix}\varphi =\begin{pmatrix}2 \\ 2 \end{pmatrix} =\operatorname{D}^2\Psi(\lambda,x)(\varphi,\varphi)\]
\[\Rightarrow \langle \operatorname{D}^2 \Psi(\lambda,x)(\varphi,\varphi),\tilde{\varphi}\rangle =\langle (2,2),(1,0)\rangle=2>0
\]
$\Rightarrow$ Die station"are Verzweigung bei $\lambda_c=0, x=0$ ist \emph{transkritisch}
\end{beispiel}

\subsection{Weiterer Instabilit"atsmechanismus f"ur Hom-Systeme}
Wir haben bereits festgestellt, dass falls ein Eigenwert die komplexe Einheitskugel über $(1,0)$ von innen nach au"sen verl"asst, dann findet dort ein Stabilit"atswechsel statt. Dies konnte analog zum kontinuierlichem Fall hergeleitet werden. Nun ist es aber f"ur Hom-Systeme auch m"oglich, dass ein Eigenwert die komplexe Einheitskugel "uber $(-1,0)$ verl"asst. Das ist f"ur kontinuierliche Systeme nicht m"oglich. Wir werden diesen neuen Fall nun untersuchen. 


Sei ein algebraisch und geometrisch einfacher Eigenwert $\gamma(\lambda)$ von $\operatorname{J}_x\Psi(\lambda,x_t)$ gegeben, der bei $\lambda=\lambda_C$ "uber $-1$ vom Inneren des Einheitskreises in der komplexen Ebene nach au"sen wandert. Dabei seien alle anderen Eigenwerte strikt im Inneren des Einheitskreises f"ur $\lambda$ nahe $\lambda_C$. Es gilt f"ur diesen Eigenwert $\gamma(\lambda)$ also
\[\gamma(\lambda_C)=-1,  \ \gamma'(\lambda_C)<0\]
Nach unserer obigen Theorie gilt somit 
\begin{align*}
x=x_t\begin{cases} \text{asymptotisch stabil f"ur } \lambda<\lambda_C\\
\text{instabil f"ur } \lambda>\lambda_C\end{cases}
\end{align*}

\begin{bemerkung}
Es kann bei $\lambda=\lambda_C$ keine station"are Verzweigung von $(\lambda,x_t)$ auftreten, da die notwendige Bedingung 
\[\operatorname{J}_x\Psi(\lambda,x_t)-\operatorname{id}\text{ singul"ar}\]
nicht erf"ullt ist.
\end{bemerkung}

\begin{satz}[im Eindimensionalem]
Sei $G\subseteq\RR\times\RR$ ein Grundl"osungszweig, $(\lambda_C, x_t) \in G$. Sei $U \subseteq \RR$, sodass $U\times\RR \subseteq G$ gilt. Weiter sei $\gamma(\lambda)$ ein algebraisch und geometrisch einfacher Eigenwert von $\operatorname{J}_x\Psi(\lambda, x_t)$. Es gelte
$$ \gamma(\lambda_C) = -1, \ \gamma'(\lambda_C) < 0$$
Sei $\lambda^*(x)$ der zugeh"orige nicht-triviale L"osungszweig.\\
Dann sind alle $(x, \lambda^*(x))$ $2$-periodische Orbits des Hom-Systems. Die Stabilit"at der Orbits ist 
\begin{itemize}
\item orbital asymptotisch stabil, falls $(x-x_t){\lambda^*}'(x) > 0$
\item instabil, falls $(x-x_t){\lambda^*}'(x) < 0$
\end{itemize}
\end{satz}

\begin{beweis}
Wir betrachten das durch $\Phi(\lambda,x):=\Psi(\lambda,\Psi(\lambda,x))$ induzierte diskrete System. Da $(\lambda_C, x_t)$ ein Gleichgewichtspunkt (bzw. im diskreten Fall ein Fixpunkt) ist, gilt
\[\forall\lambda\in U: \Phi(\lambda,x_t)=\Psi(\lambda,\underbrace{\Psi(\lambda,x_t)}_{=x_t})=x_t\ \]
Dann ist $(\lambda,x_t), \lambda\in U$ ein Grundl"osungszweig trivialer Gleichgewichtspunkte für das $\Phi$-System. F"ur die Jacobi-Matrix gilt
\[\operatorname{J}_x\Phi(\lambda,x)=\operatorname{J}_x\Psi(\lambda,\Psi(\lambda,x))\operatorname{J}_x\Psi(\lambda,x))\]
\[\Rightarrow \operatorname{J}_x\Phi(\lambda,x_t)=\operatorname{J}_x\Psi(\lambda,\Psi(\lambda,x_t))\operatorname{J}_x\Psi(\lambda,x_t)=(\operatorname{J}_x\Psi(\lambda,x_t))^2\]
Daher ist $\gamma^2(\lambda)$ ein algebraisch und geometrisch einfacher Eigenwert von $\operatorname{J}_x\Phi(\lambda,x_t)=(\operatorname{J}_x\Psi(\lambda,x_t))^2$, mit den gleichen Eigenvektoren.
Es gilt
\[\gamma^2(\lambda_C)=(-1)^2=1\]
\[\frac d {d\lambda}\gamma^2(\lambda)=2\gamma(\lambda)\gamma'(\lambda)\]
\[\left. \frac d {d\lambda}\gamma^2(\lambda)\right |_{\lambda=\lambda_C}=2\underbrace{\gamma(\lambda_C)}_{=-1}\underbrace{\gamma'(\lambda_C)}_{<0}>0\]
Daher ist $(\lambda_C,x_t)$ ein station"arer Verzweigungspunkt des $\Phi$-Systems. Jeder Fixpunkt (Gleichgewichtspunkt) von $\Phi$ nahe $\lambda=\lambda_C, x=x_t$ ist ein $2$-periodischer Punkt von $\Psi(\lambda,\cdot)$. Diese tauchen paarweise auf und bilden jeweils einen $2$-periodischen Orbit des $\Psi(\lambda,\cdot)$-Systems, d.h. $(\lambda_C,x_t)$ ist ein Verzweigungspunkt, bei dem, vom Grundl"osungszweig trivialer Gleichgewichtspunkte, ein Zweig $2$-periodischer Orbits des $\Phi$-Systems abzweigt. Insbesondere gilt auch (unter den gemachten Voraussetzungen) das Prinzip des Stabilit"atsaustausches im Sinne der orbitalen (asymptotischen) Stabilit"at bzw. Instabilit"at der $2$-periodischen Orbits
\qed
\end{beweis}
Analog kann obiger Satz auch f"ur Systeme in beliebiger Dimension formuliert werden.

%% VL 23.12.13

\section{Hopf-Verzweigung f"ur kontinuierliche Systeme}


\begin{satz}[Hopf-Verzweigung f"ur ebene Systeme]
Sei $v\in C^k(\RR\times\RR^2, \RR^2)$ f"ur ein $k\geq 4$. Sei $G\subseteq \RR\times\RR^2$ ein Grundl"osungszweig. Es existiere ein $(\lambda_C, x_C) \in G$ sodass die Jacobi-Matrix $\operatorname{J}_x v(\lambda, x)$ f"ur alle $(\lambda, x_C)$ in einer Umgebung $U$ von $(\lambda_C, x_C)$ ein Paar komplex konjugierte Eigenwerte $\gamma_\pm(\lambda) = \alpha(\lambda) \pm i \beta(\lambda)$ besitzt. F"ur diese Eigenwerte gelte
$$ \beta(\lambda_C) > 0, \ \alpha(\lambda_C) = 0, \ \alpha'(\lambda_C) > 0 $$
Dann gilt
\begin{enumerate}
\item Es existiert $\varepsilon > 0, \ \lambda = \lambda^*(x_1) \in C^{k-3}((-\varepsilon, \varepsilon))$, sodass f"ur jedes $x_1 - x_C^1 > 0, \ |x_1 - x_C^1| < \varepsilon$ ein periodischer Orbit von $\dot x = v(\lambda^*(x_1), x)$ durch den Punkt $(x_1 + x_C^1, x_C^2) \in \RR^2$ existiert, welcher den Punkt $x_C$ uml"auft.
\item $(\lambda_C, x_C)$ ist ein Verzweigungspunkt. Die trivialen Gleichgewichtspunkte sind die einzigen Gleichgewichtspunkte in einer Umgebung von $(\lambda_C, x_C)$. Gleiches gilt f"ur die periodischen Orbits.
\item Es gilt das \emph{Prinzip des Stabilit"atsaustausches} zwischen trivialen Gleichgewichtspunkten und den Abzweigungen periodischer Orbits.
\item Sei $v(\lambda, x) = \begin{pmatrix} v_1(\lambda, x) \\ v_2(\lambda, x) \end{pmatrix}$, wobei die Koordinaten $x_1, x_2$ so gew"ahlt sind, dass $\operatorname{J}_xv$ in reeller Jordan-Normalform ist f"ur $\lambda = \lambda_C$
$$ \operatorname{J}_x v(\lambda_C, x_C) = \begin{pmatrix}
\alpha(\lambda_C)  &\beta(\lambda_C) \\ -\beta(\lambda_C)& \alpha(\lambda_C)
\end{pmatrix}$$
Setze 
\begin{align*}
\begin{split}
 a &= \left (\frac {1}{16}\left[\vphantom{\frac{1}{16}}\partial_{x_1x_1x_1}v_1 + \partial_{x_1x_2x_2}v_1 + \partial_{x_1x_1x_2}v_2 + \partial_{x_2x_2x_2}v_2\right] \right . \\
 &+ \frac{1}{16\beta}\left[ \vphantom{\frac{1}{16}} \partial_{x_1x_2}v_1\left(\partial_{x_1x_1}v_1 + \partial_{x_2x_2}{v_1}\right) - \partial_{x_1x_2}v_2\left(\partial_{x_1x_1}v_2 + \partial_{x_2x_2}v_2\right) \right .\\
 &- \left . \left . \partial_{x_1x_1}v_1\partial_{x_1x_1}v_2 + \partial_{x_2x_2}v_1\partial_{x_2x_2}v_2 \vphantom{\frac{1}{16}} \right] \right )(\lambda_C, x_C)
\end{split} 
\end{align*}
Dann ist $(\lambda_C, x_C)$ 
\begin{itemize}
\item superkritisch und die periodischen Orbits sind orbital asymptotisch stabil, falls $a>0$.
\item subkritisch und die periodischen Orbits sind orbital instabil, falls $a<0$.
\end{itemize}
\end{enumerate}
\end{satz}



\end{document}
